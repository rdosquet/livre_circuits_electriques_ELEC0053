% !TeX root = syllabus_ELEC0053.tex
% !TeX encoding = ISO-8859-1
% !TeX spellcheck = fr_FR

\section{Exercices}

\begin{exercise}{Bilan de puissance}
\label{ex:1-1}
%\paragraph{Exercice 1.1 - Bilan de puissance.}
Déterminer le bilan de puissance pour chacun des 3 circuits ci-dessous.
\begin{center}
\includegraphics{exercices/ex-1-1}
\end{center}
\rep{Circuit 1: $p_R=\frac{4}{3}$ W, $p_J=2$ W, $p_E=-\frac{2}{3}$ W \\
circuit 2: $p_R=3$ W, $p_J=5$ W, $p_E=-2$ W \\ 
circuit 3: $p_{1\Omega}=4$ W, $p_{3\Omega}=3$ W, $p_E=2$ W, $p_J=5$ W.}
\end{exercise}

\begin{exercise}{Lois de Kirchhoff et loi d'Ohm.}
\label{ex:1-2}
Déterminer le courant $I_0$ débité par la source $V_s$ du
circuit ci-dessous à l'aide des lois de Kirchhoff et de
la loi d'Ohm.
\begin{center}
	\includegraphics[width=0.8\textwidth]{exercices/ex-1-2}
\end{center}
\rep{$I_0=-3 A$}
\end{exercise}

\begin{exercise}{Association de résistances.}
\label{ex:1-3}
Déterminer la résistance équivalente 
\begin{enumerate}
	\item à l'association en série
	des $n$ résistances $R_1$, $R_2$, \ldots, $R_n$.
	\item à l'association en parallèle
	des $n$ résistances $R_1$, $R_2$, \ldots, $R_n$.
	\item du dipôle de la figure ci-dessous
	à l'aide de réductions successives d'associations d'éléments en série
	ou en parallèle.
\end{enumerate}
\begin{center}
	\includegraphics[width=0.8\textwidth]{exercices/ex-1-3}
\end{center}

\rep{1) $R_{eq}=R_1+R_2+\ldots +R_n$ \\
	2) $G_{eq}=G_1+G_2+\ldots +G_n$ \\
	3) $R_{eq} = 12\,\, \Omega$}
\end{exercise}

\begin{exercise}{Le diviseur potentiométrique.}
\label{ex:1-4}
\begin{enumerate}
	\item Déterminer l'expression des tensions $V_1$ et $V_2$ aux bornes
	des résistances $R_1$ et $R_2$ du diviseur potentiométrique de la
	figure ci-dessous.
	\item On alimente une résistance de charge $R_L$ par le tension
	$V_2$. Déterminer sous quelle condition la valeur de $V_2$ est peu
	influencée par la résistance de charge $R_L$.
\end{enumerate}
\begin{center}[h]
	\includegraphics[width=\textwidth]{exercices/ex-1-4}
\end{center}
\end{exercise}


\begin{exercise}{Le diviseur de courant.}
\label{ex:1-5}
Déterminer l'expression des courants $I_1$ et $I_2$ parcourant les
résistances $R_1$ et $R_2$ de la figure ci-dessous.
\begin{center}
	\includegraphics[width=0.55\textwidth]{exercices/ex-1-5}
\end{center}
\end{exercise}


\begin{exercise}{Sources réelles de tension et de courant.}
\label{ex:1-6}
\begin{enumerate}
\item Montrer qu'une source réelle de tension alimentant une résistance de
charge $R_L$ se comporte comme une source idéale si $R_i \ll R_L$, cf. Figure~\ref{fig:source_de_tension_reelle}.
\item Montrer qu'une source réelle de courant alimentant une résistance de
charge $R_L$ se comporte comme une source idéale si $G_i \ll G_L$, cf. Figure~\ref{fig:source_courant}.
\end{enumerate}


\rep{Voir Section~\ref{sec:sie}}
\end{exercise}

\begin{exercise}{Équivalence de sources.}
\label{ex:1-7}
\begin{enumerate}
\item Montrer, comme indiqué à la figure suivante, qu'un dipôle "source de tension" peut être
remplacé par un dipôle équivalent "source de courant" si $I_s = \frac{V_s}{R_s}$ et $G_s = \frac{1}{R_s}$.
\begin{center}
	\includegraphics[width=0.5\linewidth]{exercices/ex-1-7-1}
\end{center}
\item Montrer, comme indiqué à la figure suivante,  qu'inversement, un
dipôle "source de courant" peut être remplacé par un dipôle
équivalent "source de tension" si $V_s = \frac{I_s}{G_s}$ et $R_s = \frac{1}{G_s}$.
\begin{center}
	\includegraphics[width=0.5\linewidth]{exercices/ex-1-7-2}
\end{center}
\item Appliquer ces transformations pour réduire le dipôle ci-dessous à un
dipôle équivalent "source de tension" ou "source de courant".
\begin{center}
	\includegraphics[width=0.45\textwidth]{exercices/ex-1-7-3}
\end{center}
\end{enumerate}
\end{exercise}

\begin{exercise}{Équivalence de sources et réduction.}
\label{ex:1-8}
\begin{enumerate}
	\item Réduire le circuit de la figure ci-dessous à un dipôle
	équivalent comprenant une source de tension $V_{eq}$ en série avec une résistance $R_{eq}$
	(dipôle ``source de tension équivalente'').
	\item Si on connecte à l'accès  11$^{'}$ une résistance de charge  $R_L=10\,
	\Omega$, calculer la puissance absorbée par $R_L$ ainsi que l'état
	électrique complet du circuit.
\end{enumerate}

\begin{center}
	\includegraphics[width=0.75\linewidth]{exercices/ex-1-8}
\end{center}


\rep{1) $V_{eq}= 13.125$ V, $R_{eq}=3.75\,\,\Omega$ \\
2) $p_{R_L}=9.112$ W}
\end{exercise}

\begin{exercise}{Source commandée.}
\label{ex:1-9}
Calculer la valeur de la source de tension $V_s$ de la figure ci-dessous si le courant
$I_{\phi}$ est égal à 5 A.
\begin{center}
	\includegraphics[width=0.6\linewidth]{exercices/ex-1-9}
\end{center}

\rep{$V_s=50$ V}
\end{exercise}

\section{Solution des exercices}

\paragraph{Exercice~\ref{ex:1-1}}~\\
{\bf Circuit 1.} Choisissons le sens des courants $I$, $I_E$ et de la tension $U$ comme indiqué ci-dessous.
\begin{center}
	\includegraphics[width=0.5\linewidth]{sol_exercices/ex1-1-1}
\end{center}
Vu la présence de la source de tension, on a directement $U$ = 2 V.
La loi d'Ohm aux bornes d'une résistance s'écrit :\\[3mm]
\begin{minipage}[c]{0.5\textwidth}
	\begin{center}
		avec les sens conventionnels de référence\\[3mm]
		\begin{center}
			\includegraphics[width=0.4\linewidth]{sol_exercices/ex1-1-2}
		\end{center}
	\end{center}
	\[U=R.I\]
\end{minipage}
\begin{minipage}[c]{0.5\textwidth}
	\begin{center}
		avec les sens non conventionnels de référence\\[3mm]
		\begin{center}
			\includegraphics[width=0.3\linewidth]{sol_exercices/ex1-1-3}
		\end{center}
	\end{center}
	\[U=-R.I \]
\end{minipage}\\[3mm]
L'application de cette loi à la résistance $R=3\, \, \Omega$ donne
directement :
\[I=\frac{U}{3}=\frac{2}{3}\text{~A}\]
La première loi de Kirchhoff (PLK) au noeud a s'écrit :
\[I+I_E-1=0 \qquad \text{on déduit} \qquad
I_E=1-I=\frac{1}{3}\text{~A}\]
L'état électrique complet du circuit est connu. On en déduit les
puissances consommées ou fournies par les différents éléments. 
\begin{enumerate}
	\item Résistance $R=3\,\, \Omega$ : les sens adoptés pour $U$ et $I$
	sont les sens conventionnels de référence
	\[p_{R}=UI=RI^2=\frac{4}{3}\text{~W}\]
	est la puissance consommée par cette résistance.
	\item Source $J=1$ A : 
	
	\begin{minipage}[c]{3cm}
		\begin{center}
			\includegraphics[width=0.6\linewidth]{sol_exercices/ex1-1-4}
		\end{center}
	\end{minipage}
	\begin{minipage}[c]{7cm}
		le courant et la  tension aux bornes de cet élément sont orientés
		selon les sens de référence non conventionnels,
	\end{minipage}
	\[p_{J}=U.1=2\text{~W}\]
	est la puissance fournie par la source $J$.
	\item Source $E=2$ V : 
	
	\begin{minipage}[c]{3cm}
		\begin{center}
			\includegraphics[width=0.6\linewidth]{sol_exercices/ex1-1-5}
		\end{center}
	\end{minipage}
	\begin{minipage}[c]{7cm}
		le courant et la tension aux bornes de cet élément sont orientés
		selon les sens de référence conventionnels,
	\end{minipage}
	\[p_{E}=U.I_E=\frac{2}{3}\text{~W}\]
	est la puissance consommée par la source. La valeur trouvée est
	positive, cette source consomme effectivement de la puissance.
	
	Si l'on avait adopté les sens non conventionnels de référence, on
	aurait :\\[3mm]
	\begin{minipage}[c]{3cm}
		\begin{center}
			\includegraphics[width=0.6\linewidth]{sol_exercices/ex1-1-6}
		\end{center}
	\end{minipage}
	\begin{minipage}[c]{7cm}
		$I^{'}_E=-I_E$ et 
		$p^{'}_{E}=U.I^{'}_E=-\frac{2}{3}\text{~W}$
		est la puissance fournie par la source. Elle est négative indiquant
		que $E$ consomme  de la puissance. Quels que soient les sens adoptés
		pour $U$ et $I$, on constate que les résultats sont cohérents.
	\end{minipage}
\end{enumerate}
Le bilan de puissance s'établit comme suit :
\[\Sigma \text{p consommées}=0 \qquad p_E+p_R-p_J=0\]
ou
\[\Sigma \text{p consommées}=\Sigma \text{p fournies} \qquad
p_E+p_R=p_J\]


\noindent{\bf  Circuit 2.}

\begin{minipage}[c]{5cm}
	\begin{center}
		\includegraphics[width=\linewidth]{sol_exercices/ex1-1-7}
	\end{center}
\end{minipage}
\begin{minipage}[c]{5cm}
	Choisissons le sens des courants $I$, $I_E$ et des  tensions $U$ et $U_J$
	comme indiqué ci-contre.
	
	Vu la présence de la source de courant, on a directement $I$ = -1 A.
	
	Par application de la loi d'Ohm : $U=3.I=-3$ V.
	
	La seconde loi de Kirchhoff (SLK) s'écrit : \\$U+U_J-2=0 \, \Rightarrow\,
	U_J=$ 5 V.
\end{minipage}
On calcule les puissances relatives aux différents éléments :
\begin{enumerate}
	\item $p_R=RI^2= 3$ W, est la puissance consommée par la résistance de
	$3\,\, \Omega$.
	\item $p_J=1.U_J=$ 5 W est la puissance fournie par la source $J$.
	\item $p_E=2.I=-2$ W est la puissance fournie par $E$. Puisque la
	valeur est négative, $E$ est en fait consommatrice de puissance.
\end{enumerate}
Le bilan de puissance s'écrit :
\[p_R=p_J+p_E\]


\noindent{\bf Circuit 3.}

\begin{center}
	\includegraphics[width=0.6\linewidth]{sol_exercices/ex1-1-8}
\end{center}
On trouve :
\begin{gather*}
U_1=2\,\, \text{V}\, \quad U_3=-3\,\,\text{V},\quad
U_J=5\,\,\text{V}\\
I_1=2\,\, \text{A}\, \quad I_3=-1\,\, \text{A}\, \quad
I_E=1\,\,\text{A}
\end{gather*}
Puissances consommées par les résistances :
\[p_{1\Omega}=R_1I_1^2=4\,\, \text{W},\quad p_{3\Omega}=R_3I_3^2=3\,\,
\text{W}\]
Puissances fournies par les sources :
\[p_E=U_1.I_E=2 \,\, \text{W},\quad p_J=U_J.1=5\,\, \text{W}\]
Bilan de puissance :
\[p_{1\Omega}+p_{3\Omega}=p_E+p_J\]

\paragraph{Exercice~\ref{ex:1-2}}~\\%
Adoptant les sens des courants et tensions de la
Fig.~\ref{ex1-1sol}, on trouve successivement:
\begin{enumerate}
	\item par application de la PLK  au noeud b ou c :
	\[ I_1-I_0-6=0\]
	\item par  application de la SLK à la maille abca :
	\[120 - V_0 - V_1 = 0\]
	\item ou en appliquant la loi d'Ohm : 
	\[120 - 10I_0 - 50I_1 =0 \]
\end{enumerate}
On trouve finalement : $I_0=-3$ A et $I_1 = 3$ A.
\begin{figure}[h]
	\centering
	\includegraphics[width=0.7\linewidth]{sol_exercices/ex1-2-1}
	\caption{Lois de Kirchhoff et loi d'Ohm.}\label{ex1-1sol}
\end{figure}
On vérifie le bilan de puissance : 
\begin{enumerate}
	\item puissance consommée par la résistance de 10 $\Omega$ : $$p_{10
		\Omega}= 10 I_0^2= 90~\text{W}$$
	\item puissance consommée par la résistance de 50 $\Omega$ : $$p_{50
		\Omega}= 50 I_1^2= 450~\text{W}$$
	\item puissance produite par la source de tension de 120 V : $$p_{120
		V}= 120 I_0= - 360~\text{W}$$
	cette source absorbe de la puissance !
	\item puissance produite par la source de courant de 6 A : $$p_{6A}
	= 6 V_1= 6.50.I_1 = 900~\text{W}$$
\end{enumerate}
On vérifie bien que la puissance totale produite par les
deux sources = puissance totale consommée par les résistances: $$p_{10\Omega}+p_{50\Omega}=p_{120V}+p_{6A}.$$

\paragraph{Exercice~\ref{ex:1-3}}~\\%
{\bf Equivalence entre deux dipôles :}
{\em deux dipôles sont équivalents si pour  une même
	d.d.p aux bornes $v$, ces dipoles sont parcourus par un même courant
	$i$ et inversement si, pour un même courant injecté $i$, il apparaît une
	même d.d.p. $v$ aux bornes des deux dipôles.}

On ne change rien à l'état électrique d'un circuit si l'on
remplace un des dipôles le constituant par un dipôle équivalent.

Nous appliquons ici ce principe au cas particulier de dipôles résistifs.

\noindent {\bf 1. Association en série}\\
Alimentées par une source de tension $V_s$, les résistances $R_1$,
$R_2$, \ldots , $R_n$ connectées en série sont parcourues par le même
courant $I_s$.
\newpage

Nous recherchons une résistance équivalente $R_{eq}$ telle que 
\begin{center}
\includegraphics[width=0.9\linewidth]{sol_exercices/ex1-3-1}
\end{center}
La SLK et la loi d'Ohm appliquées dans la maille fournissent :
\[V_s=R_1I_s+R_2I_s+\ldots + R_nI_s\quad \text{et}\quad
I_s=\frac{V_s}{R_1+R_2+\ldots + R_n}=\frac{V_s}{R_{eq}}\]
Finalement la résistance équivalente est donnée par la somme des $n$
résistances
\[R_{eq}=R_1+R_2+\ldots +R_n\]

\noindent {\bf 2. Association en parallèle}\\
Il existe une même d.d.p. aux bornes de tous les éléments.
\begin{center}
	\includegraphics[width=0.9\linewidth]{sol_exercices/ex1-3-2}
\end{center}
On écrit :
\[V_s=R_1\, I_1 = R_2\, I_2 = \ldots = R_n\, I_n \]
\[I_s=I_1+I_2+ \ldots + I_n= V_s(\frac{1}{R_1}+\frac{1}{R_2}+ \ldots +
\frac{1}{R_n})\]
On peut représenter ce dipôle par une résistance équivalente qui vérifie :
\[\frac{I_s}{V_s}=\frac{1}{R_{eq}}=\frac{1}{R_1}+\frac{1}{R_2}+\ldots
+  \frac{1}{R_n}\]
ou en termes de conductances :
\[G_{eq}=G_1+G_2+ \ldots + G_n\]

\newpage
\noindent {\bf 3.}
On calcule successivement :
\begin{center}
	\includegraphics[width=0.9\linewidth]{sol_exercices/ex1-3-3}
\end{center}

\paragraph{Exercice~\ref{ex:1-4}}~\\%
\noindent{\bf 1. Expression de $V_1$ et $V_2$}
On dérive successivement :
\begin{align*}
I&=\frac{V_s}{R_1+R_2}\\
V_1&=R_1\,I=V_s\frac{R_1}{R_1+R_2}\\
V_2&=R_2\,I=V_s\frac{R_2}{R_1+R_2} 
\end{align*}

Le choix de $R_1$ et$R_2$ permet de fixer la manière dont $V_s$ est
répartie entre $V_1$ et $V_2$. Il existe une infinité de couples de valeurs $(R_1,R_2)$
donnant lieu à la même répartition. Le choix des valeurs de ces deux
résistances est aussi guidé par :
\begin{enumerate}
	\item la puissance qui peut être dissipée par chaque élément;
	\item la résistance de charge éventuelle que doit alimenter $V_1$ ou
	$V_2$ comme montré au point~2.
\end{enumerate}
\noindent{\bf 2. Influence de $R_L$}
On dérive successivement :
\begin{align*}
V_2&=\frac{R_{eq}}{R_{eq}+R_1}V_s \quad \text{avec} \quad 
R_{eq}=\frac{R_2 R_L}{R_2 + R_L}\\
V_2 &= \frac{R_2}{R1[1+ (R_2/R_L)]+R_2}V_s
\end{align*}
Nous aurons donc 
\[\frac{V_2}{V_s}\simeq \frac{R_2}{R_1+R_2} \quad \text{si} \quad R_L \gg R_2.\]

\paragraph{Exercice~\ref{ex:1-5}}~\\%
La tension aux bornes des deux résistances est donnée par 
\[V=I_s\frac{R_1 R_2}{R_1+R_2}\]
Le courant $I_s$ se divise donc comme suit :
\begin{align*}
I_1 &=\frac{V}{R_1}=I_s\frac{R_2}{R_1+R_2} \quad \text{ou} \quad
I_s\frac{G_1}{G_1+G_2}\\
I_2 &=\frac{V}{R_2}=I_s\frac{R_1}{R_1+R_2} \quad \text{ou} \quad
I_s\frac{G_2}{G_1+G_2}
\end{align*}

\paragraph{Exercice~\ref{ex:1-6}}
Voir Exercice précédent et /ou la Section~\ref{sec:sie}.

\paragraph{Exercice~\ref{ex:1-7}}~\\%
\noindent{\bf 1. Equivalence source de tension $\rightarrow$ source de
	courant.}
Les deux dîpôles sont équivalents si ils délivrent un même courant $I$
sous une même tension $V$.
Le courant débité par le dipôle ``source de tension'' s'écrit :
	\[I= \frac{V_s}{R_s}-\frac{V}{R_s}\]
Le courant débité par le dipôle ``source de courant'' s'écrit :
	\[I=I_s-G_s V\]
La condition d'équivalence s'écrit donc :
\[\frac{V_s}{R_s}-\frac{V}{R_s}=I_s-G_s V\]
Les paramètres du dipôle ``source de courant'' équivalent sont :
\begin{align*}
I_s&=\frac{V_s}{R_s}\\
G_s&=\frac{1}{R_s}
\end{align*}

\noindent{\bf 2. Equivalence source de courant $\rightarrow$ source de
	tension.}
Les deux dipôles sont équivalents si ils présentent une même tension $V$
sous un même courant $I$.
La condition d'équivalence s'écrit donc :
\[\frac{I_s}{G_s}-\frac{I}{G_s}=V_s-R_s I\]
Les paramètres du dipôle ``source de tension'' équivalent sont :
\begin{align*}
V_s&=\frac{I_s}{G_s}\\
R_s&=\frac{1}{G_s}
\end{align*}


\noindent{\bf 3. Exemple}
On transforme successivement le circuit comme suit.
\begin{center}
	\includegraphics[width=\linewidth]{sol_exercices/ex1-7}
\end{center}

\paragraph{Exercice~\ref{ex:1-8}}~\\%
\noindent{\bf 1. Source de tension équivalente.}

On transforme successivement :
\begin{enumerate}
	\item la branche CB\\
	\includegraphics[width=0.4\linewidth]{sol_exercices/ex1-8-1}
	\item la branche DE\\
	\includegraphics[width=\linewidth]{sol_exercices/ex1-8-2}
\end{enumerate}
La partie du circuit située au-dessus de AE se réduit ainsi en la
forme
\begin{center}
	\includegraphics[width=\linewidth]{sol_exercices/ex1-8-3}
\end{center}
Connectant au reste du circuit, on a 
\begin{center}
	\includegraphics[width=\linewidth]{sol_exercices/ex1-8-4}
\end{center}

Finalement
\[V_{eq}=13.125\,\, \text{V}, \quad R_{eq}=3.75\,\, \Omega\]

\noindent{\bf 2. Accès 11$^{'}$ fermé sur $R_L=10\,\, \Omega$.}

Le circuit étant remplacé par son dipôle equivalent ``source de tension'',  on connecte
la résistance de charge à l'accès 11$^{'}$.\\
\begin{minipage}[c]{0.3\textwidth}
	\includegraphics[width=\linewidth]{sol_exercices/ex1-8-5}
\end{minipage}
\begin{minipage}[c]{0.7\textwidth}
	On déduit le courant débité dans la résistance de charge :
	\[I=\frac{V_{eq}}{R_{eq}+R_L}=0.9545\,\, \text{A}\]
	La tension aux bornes de cette charge vaut :
	\[U=RI=10I=9.545\,\, \text{V}\]
	La puissance consommée par la charge vaut :
	\[p_{R_L}=R_L I^2=9.112\,\, \text{W}\]
\end{minipage}

\noindent{\bf 3. Etat électrique complet du circuit}

Dans ce qui suit, les puissances calculées relatives aux résistances
sont les puissances consommées par ces résistances. Les puissances calculées
relatives aux sources de tension et de courant sont les puissances
fournies par ces sources au reste du circuit.

\begin{enumerate}
	\item branche AE :
	
	\begin{minipage}[c]{5cm}
		\begin{center}
			\includegraphics[width=0.7\linewidth]{sol_exercices/ex1-8-6}
		\end{center}
	\end{minipage}
	\begin{minipage}[c]{5cm}
		\begin{align*}
		I^{'}&=\frac{U}{5}=1.909\,\, \text{A}\\
		I_A&=I+I^{'}-3=-0.1365\,\, \text{A}\\\
		p_{J=3}&=3U=28.635\, \, \text{W}\\
		p_{R=5}&=5I^{'2}=18.22 \,\, \text{W}
		\end{align*}
	\end{minipage}
	\item dipôle AB :
	
	\begin{minipage}[c]{5cm}
		\begin{center}
			\includegraphics[width=0.4\linewidth]{sol_exercices/ex1-8-7}
		\end{center}
	\end{minipage}
	\begin{minipage}[c]{5cm}
		\begin{align*}
		p_{E=10}=10I_A = -1.365\, \, \text{W}
		\end{align*}
		Cette source consomme de la puissance.
	\end{minipage}
	\item dipôle BC :
	
	\begin{minipage}[c]{5cm}
		\begin{center}
			\includegraphics[width=0.55\linewidth]{sol_exercices/ex1-8-8}
		\end{center}
	\end{minipage}
	\begin{minipage}[c]{5cm}
		\begin{align*}
		U_{CB}&=5(I_A-2)= -10.682\, \, \text{A}\\
		p_{R=5}&=\frac{U_{CB}^2}{5}=22.823\, \, \text{W}\\
		p_{J=2}&=-2U_{CB}=21.365\, \, \text{W}
		\end{align*}
	\end{minipage}
	\item dipôle CD :
	
	\begin{minipage}[c]{5cm}
		\begin{center}
			\includegraphics[width=0.55\linewidth]{sol_exercices/ex1-8-9}
		\end{center}
	\end{minipage}
	\begin{minipage}[c]{5cm}
		\begin{align*}
		p_{R=5}=5I_A^2=0.093\, \, \text{W}
		\end{align*}
	\end{minipage}
	
	
	\item dipôle DE :
	
	\begin{minipage}[c]{5cm}
		\begin{center}
			\includegraphics[width=\linewidth]{sol_exercices/ex1-8-10}
		\end{center}
	\end{minipage}
	\begin{minipage}[c]{5cm}
		\begin{align*}
		U_{DE}&=U-10+U_{CB}+U_{DC}= -11.82\, \, \text{V}\\
		p_{J=2}&=-2U_{DE}=23.635\, \, \text{W}\\
		p_{R_1=10}&=\frac{U_{DE}^2}{10}= 13.971\, \, \text{W}\\
		p_{R_2=10}&=\frac{(U_{DE}+5)^2}{10}= 4.648\, \, \text{W}\\
		I_5&=\frac{U_{DE}+5}{10}=-0.682\, \, \text{V}\\
		p_{E=5}&=5I_5= -3.409\, \, \text{W}
		\end{align*}
		Cette source consomme de la puissance.
	\end{minipage}
\end{enumerate}


\paragraph{Exercice~\ref{ex:1-9}}
\begin{center}
	\includegraphics[width=0.7\linewidth]{sol_exercices/ex1-9}
\end{center}
La PLK appliquée au noeud a et b fournit :
\begin{align*}
I_s&=I_{\phi}+5 = 10\\
I_s&=6I_{\phi} + I_1 \quad \rightarrow \quad I_1 = I_s-6I_{\phi}=-20
\end{align*}
La SLK appliquée à la maille abca s'écrit :
\[\begin{array}[t]{ll}
V_s+V_1+V_2=0 \quad \text{avec} \quad & V_2=10I_{\phi} = 50\\
& V_1 = 5I_1= -100 \end{array}\]
Finalement : $V_s=50$ V.
