% !TeX root = syllabus_ELEC0053.tex
% !TeX encoding = ISO-8859-1
% !TeX spellcheck = fr_FR

\section{Exercices}
\begin{exercise}{Méthode des noeuds}\label{ex:mg-1}
Déterminer l'état électrique complet du réseau de la  figure ci-dessous au
moyen de la méthode des noeuds.
\begin{center}
	\includegraphics[width=0.7\textwidth]{exercices/ex-1-18}
\end{center}
\rep{${\bf V}_N=(-0.112, 0.306, 0.395, 0.322)^T$ V\\ noeud de référence
	: e}
\end{exercise}

\begin{exercise}{Méthode des mailles}\label{ex:mg-2}
Déterminer l'état électrique complet du réseau de la  figure ci-dessous au
moyen de la méthode des mailles.
\begin{center}
	\includegraphics[width=0.7\textwidth]{exercices/ex-1-19}
\end{center}

\rep{${\bf I}_M=(0.224,\, 0.967,\, -0.612,\, -0.111,\, 1.479)^T$ A\\
	arbre : 3($R=1/4$, $E=1/4$)-5($R=1/3$)-6($R=1/2$)-7($R=1/10$)}
\end{exercise}

\begin{exercise}{Méthode des mailles - réduction du nombre d'accès}\label{ex:mg-3}
Déterminer l'état électrique complet du réseau de la  figure ci-dessous.
Déterminer tout d'abord la matrice de résistances réduite
aux accès vue des deux sources indépendantes de tension.
\begin{center}
	\includegraphics[width=0.5\textwidth]{exercices/ex-1-20}
\end{center}

\rep{$(I_1,I_2)=(-0.373, 3.227)$ A}
\end{exercise}

\begin{exercise}{Méthode des noeuds - Norton}\label{ex:mg-4}
Déterminer le schéma équivalent de Norton du circuit de la
 figure ci-dessous vu de l'accès 11$^{'}$. Exprimer les paramètres de
cet équivalent en fonction de $E$. En déduire le schéma équivalent de
Thévenin vu de ce même accès.

\begin{center}
\includegraphics[width=0.6\textwidth]{exercices/ex-1-21}\\
$R_0=150\mbox{~k}\Omega \quad ; \quad R_1=100  \mbox{~k}\Omega  \quad
; \quad R_2=0.1\mbox{~k}\Omega $\\
$R_3=1000 \mbox{~k}\Omega \quad ; \quad g=0.05 \mbox{~S}$
\end{center}

\rep{$I_{No}=1.89\, 10^{-2}\, E$, $G_{No}=7.17\, 10^{-3}$ S}
\end{exercise}

\begin{exercise}{Méthode des noeuds - Norton}\label{ex:mg-5}
Déterminer le schéma équivalent de Norton du circuit
de la  figure ci-dessous vu des accès 11$^{'}$ et 22$^{'}$. 
\begin{center}
	\includegraphics[width=0.5\textwidth]{exercices/ex-1-22}
\end{center}
\rep{${\bf I}_{No}=(\frac{20}{7},\frac{40}{7})^T$ A, ${\bf G}_{No}=
\left(\begin{array}{cc}
7.43 & -6.14\\
-6.14 & 8.71
\end{array}\right)$ S}
\end{exercise}

\begin{exercise}{Choix de la bonne méthode}\label{ex:mg-6}

Soit un cube dont chaque arête est occupée par une résistance
de 1 $\Omega$ comme indiqué à la  figure ci-dessous. Déterminer :
\begin{enumerate}
	\item la résistance équivalente $R_a$ vue d'une arête;
	\item la  résistance équivalente $R_d$ vue d'une diagonale d'une face du cube.
\end{enumerate}
\begin{center}
	\includegraphics[width=0.5\textwidth]{exercices/ex-1-23}
\end{center}
\rep{$R_a=R_{18}=0.58\,\Omega,\, R_d= R_{13}=0.75\,\Omega$}
\end{exercise}

\begin{exercise}{Thévenin - méthode des mailles}\label{ex:mg-7}
Déterminer l'expression des rapports $\frac{V_5}{E}$ et $\frac{V_5}{I}$ pour le circuit de
la  figure ci-dessous. Considérer le cas général et les cas particuliers : 
\begin{enumerate}
	\item de la source idéale ($R_6=0$), 
	\item du détecteur d'impédance infinie
	($R_5=\infty$). 
\end{enumerate}
\begin{center}
	\includegraphics[width=0.35\textwidth]{exercices/ex-1-24}
\end{center}
\end{exercise}

\begin{exercise}{Méthode des mailles ou transfiguration étoile - triangle}\label{ex:mg-8}
Déterminer le schéma équivalent de Thévenin du circuit
de la  figure ci-dessous vu de l'accès 11$^{'}$. Déterminer ensuite la puissance
produite par la source de 12 V si l'accès 11$^{'}$ est laissé ouvert.
\begin{center}
	\includegraphics[width=0.6\textwidth]{exercices/ex-1-25}
\end{center}

\rep{$V_{Th}=9.71$ V, $R_{Th}=20.85\, \Omega$, $p_{12V}=2.54$ W}
\end{exercise}

\begin{exercise}{Thévenin - méthode des mailles}\label{ex:mg-9}
Déterminer la valeur de la résistance $R_0$ du circuit de la
 figure ci-dessous telle que cette résistance consomme une puissance de
1000 W.
\begin{center}
	\includegraphics[width=0.6\textwidth]{exercices/ex-1-26}
\end{center}
\rep{$R_0^{(1)}=17.14\, \Omega, R_0^{(2)}=0.37\,\Omega$}
\end{exercise}

\begin{exercise}{Méthode des noeuds - Norton}\label{ex:mg-10}
Déterminer la puissance dissipée dans la résistance $R$
connectée aux bornes 14 du circuit de la  figure ci-dessous.

Suggestion : utiliser la méthode des noeuds pour rechercher le schéma
équivalent de Norton vu des bornes 14. 
\begin{center}
\includegraphics[width=0.8\textwidth]{exercices/ex-1-27}
$G_{12}=35 \mbox{~mS}\quad ; \quad G_{13}=25 \mbox{~mS}\quad ; \quad 
G_{23}=10 \mbox{~mS}\quad ; \quad G_{24}=80 \mbox{~mS}$\\
$G_{34}=75 \mbox{~mS}\quad ; \quad g=20 \mbox{~mS}\quad ; \quad
R=100 \;\; \Omega \quad ; \quad J=5 \mbox{~A}$\\
On pose $\quad U_{14}=U_4-U_1 \quad \mbox{et}\quad U_{23}=U_3-U_2$
\end{center}
\rep{$p_R=49.16$ W}
\end{exercise}


\begin{exercise}{Méthodes générales, régime sinusoïdal}\label{ex:RSE-9}
	Déterminer l'inductance équivalente du dipôle de la
	figure ci-dessous sachant que les inductances $L_1$ et $L_4$ sont
	couplées avec un coefficient de couplage $k=0.7$.
	\begin{figure}[h]
		\begin{center}
			\includegraphics[width=0.5\linewidth]{exercices/ex-2-9}\\
			$L_1=0.1 \mbox{~mH} \,\, ; \,\, L_2=0.5 \mbox{~mH} \,\, ; \,\,
			L_3=0.2 \mbox{~mH} \,\, ; \,\, L_4=0.4 \mbox{~mH} \,\, ; \,\,
			L_5=0.8 \mbox{~mH}$\\
			\caption{}\label{ex2-9}
		\end{center}
	\end{figure}
	\rep{$L_{eq}=0.413$ mH}
\end{exercise}

\begin{exercise}{Méthodes générales, régime sinusoïdal}\label{ex:RSE-10}
	Étant donné le circuit ci-dessous fonctionnant en
	régime sinusoïdal établi :
	\begin{enumerate}
		\item déterminer l'impédance $Z_L$ à connecter aux bornes 11$^{'}$
		pour soutirer au circuit le maximum de puissance;
		\item rechercher des éléments qui pourraient réaliser cette impédance;
		\item calculer la puissance complexe fournie à la charge $Z_L$ dans
		ces conditions.
	\end{enumerate}
	
	\begin{minipage}[c]{5cm}
		\flushright \includegraphics[width=\linewidth]{exercices/ex-2-10}\\
	\end{minipage}
	\begin{minipage}[c]{5cm}
		\begin{center}
			$R_1=10\mbox{~}\Omega \quad ; \; R_2=25\mbox{~}\Omega $\\
			$R_3=50\mbox{~}\Omega \quad ; \; R_4=50\mbox{~}\Omega $\\
			$L_1=20\mbox{~mH}\quad ; \; L_2=80\mbox{~mH}$\\
			$M=10\mbox{~mH}\quad ; \;
			\omega=1000 \mbox{~rad/s}$\\
			$\bar{E}=100\angle 0^{\circ} \mbox{V, ~valeur efficace}$
		\end{center}
	\end{minipage}
	
	
	\rep{$Z_L=33.31-j3.47\,\,\Omega$ , $R_L=33.31\,\,\Omega$ , $C_L=288\,\,\mu$F\\
		$S_{Z_L}=55.4-j5.77$ VA}
\end{exercise}


\section{Exercices non résolus}

\begin{exercise}{}\label{ex:mg-11}
	Déterminer la valeur de la résistance $R$  telle que la
	résistance d'entrée $R_{in}$ du circuit de la  figure ci-dessous vue de l'accès
	11$^{'}$ soit égale à $R$.
	\begin{center}
		\includegraphics[width=0.5\textwidth]{exercices/ex-1-28}
		\[R_1=20\, \Omega \quad , \quad R_2=5\, \Omega\]
	\end{center}
	
	\rep{$R=11.55\,\,\Omega$}
\end{exercise}


\begin{exercise}{}\label{ex:mg-12}
Déterminer la valeur de la source de courant $J$ de façon à ce
que la source de tension $E=10$~V fournisse au reste du circuit de la
 figure ci-dessous une puissance de 20 W.

Établir le bilan de puissance de ce circuit.
\begin{center}
\includegraphics[width=0.7\textwidth]{exercices/ex-1-29}
\end{center}

{\em Suggestion : rechercher dans un premier temps le schéma
	équivalent de Thévenin du circuit vu des bornes AB}
\rep{$J=2.47$ A}
\end{exercise}

\begin{exercise}{}\label{ex:mg-13}
Étant donné le circuit de la  figure ci-dessous on demande  de
déterminer son schéma équivalent de Thévenin vu des accès 11$^{'}$ et
22$^{'}$.

On connecte ensuite une source de tension idéale $E=20$ V à l'accès
11$^{'}$ et une résistance de charge $R_L=30\, \Omega$ à l'accès
22$^{'}$. Calculer dans ces conditions les puissances consommée par
$R_L$ et fournie par $E$ (penser à utiliser le schéma équivalent de
Thévenin déterminé au préalable !).
\begin{center}
	\includegraphics[width=\textwidth]{exercices/ex-1-30}
\end{center}
\rep{${\bf E}_{Th}=\left(
	\begin{array}{c}
	21.65\\
	13.01
	\end{array}\right)\quad , \quad R_{Th}=\left(
	\begin{array}{cc}
	26.52& 5.13\\
	5.13 & 27.88
	\end{array}\right)$\\
	$p_{R_L}=1.49$ W, $p_E=-0.38$ W}
\end{exercise}

\begin{exercise}{}\label{ex:mg-14}
On demande d'établir le bilan de puissance du circuit
de la  figure ci-dessous sachant que  $I_d=4$ A et $I_g=2$~A.
\begin{center}
\includegraphics[width=0.7\textwidth]{exercices/ex-1-31}\\
$R_1=30\,\, \Omega ; \quad R_2=10\,\,  \Omega  
; \quad R_3=40\,\, \Omega ; \quad R_4=10 \,\, \Omega 
; \quad R_5=80  \,\, \Omega  ; \quad R_6=40  \,\, \Omega$\\
$E_1 = 60\mbox{~V}  ; \quad E_2=20\mbox{~V};\quad  J=10  \mbox{~A}$
\end{center}

${\cal R}_g$ est un circuit résistif linéaire et invariant comportant
un certain nombre de sources indépendantes d'énergie. ${\cal R}_d$ est
un circuit résistif linéaire et invariant ne comportant pas de source
indépendante d'énergie.

{\em Suggestion : en vue de calculer l'état électrique complet du
	circuit, remplacer dans un premier temps le circuit à droite de AB
	par son schéma équivalent de Norton.}

\rep{$U_{{\cal R}_g}=60$ V, $U_{{\cal R}_d}=200$ V, $U_J=523.2$ V}
\end{exercise}

\begin{exercise}{}\label{ex:mg-15}
Le circuit $\cal R$ de la  figure ci-dessous est constitu\'e d'\'el\'ements
lin\'eaires, invariants et passifs et comporte des sources
ind\'ependantes d'\'energie, notamment les sources de tension $E_1$ et
$E_2$. Si $E_1=30$ V et $E_2=10$ V, on relève $U_{11^{'}}=22$ V. Si on
double $E_1$, $U_{11^{'}}$ devient 35 V tandis que si on double $E_2$,
$U_{11^{'}}$ devient 25 V. 

On demande de calculer $E_1$ et $E_2$ pour que la résistance
$R=100\,\, \Omega$ consomme une puissance de 4 W, la somme $E_1+E_2$
devant valoir 30 V.

\begin{center}
\includegraphics[width=0.5\textwidth]{exercices/ex-1-32}
\end{center}

\rep{$E_1=37.5$ V, $E_2=-7.5$ V}
\end{exercise}

\begin{exercise}{}\label{ex:mg-16}
Étant donné le circuit de la figure ci-dessous, on cherche à
déterminer la valeur de $E$ (positive) telle que le circuit fournisse
une puissance égale à 20 W à la résistance de charge $R_L=5\,\,
\Omega$ connectée à l'accès 11$^{'}$. Pour cela :

\begin{enumerate}
	\item utiliser la méthode des mailles pour dériver le schéma
	équivalent de Thévenin vu de l'accès~11$^{'}$; 
	\item utiliser la méthode des noeuds pour dériver le schéma
	équivalent de Norton vu de l'accès~11$^{'}$. 
	
	Les paramètres de ces équivalents seront exprimés en fonction de $E$.
	\item Montrer l'équivalence des deux approches;
	\item déduire d'un des deux équivalents la valeur de $E$ cherchée.
\end{enumerate}

\begin{center}
\includegraphics[width=0.6\textwidth]{exercices/ex-1-33}
\end{center}

\rep{$E=65.7$ V}
\end{exercise}

\begin{exercise}{Quadripôles}\label{ex:mg-17}
Pour chacun des trois des quadripôles ci-dessous, déterminer une matrice de quadripôle qui le caractérise.
\begin{center}
	\includegraphics[width=0.7\textwidth]{exercices/ex-4-3}
\end{center}
\end{exercise}

\begin{exercise}{Quadripôles}\label{ex:mg-18}

Déterminer la matrice d'admittances ${\bf Y}$ du quadripôle suivant : 
\begin{center}
	\includegraphics[width=0.3\textwidth]{exercices/ex-4-4}
\end{center}
On ferme l'accès 22' sur la résistance de charge $R_L=4\,\,
\Omega$. On demande de déterminer dans ce cas la réponse fréquentielle 
gain en tension
\[H(j\omega)=\frac{\bar{U}_2}{\bar{U}_1}\]

Quelles sont les fréquences naturelles du circuit?
\rep{
	${\bf Y}= \left(
	\begin{array}{cc}
	1+j\omega & -j\omega \\ -j\omega & \frac{1}{2}+j\omega
	\end{array} \right)$\\
	$H(j\omega)=\frac{j\omega}{j\omega+\frac{3}{4}}$
}
\end{exercise}

\begin{exercise}{Quadripôles}\label{ex:mg-19}
Déterminer les valeurs des impédances images $Z_{i1}$, $Z_{i2}$
et des impédances caractéristiques $Z_{c1}$ et $Z_{c2}$ du quadripôle suivant :
\begin{center}
	\includegraphics[width=0.3\textwidth]{exercices/ex-4-5}
\end{center}
\rep{$Z_{i1}=17.3\, \Omega$, $Z_{i2}=11.5\, \Omega$\\
	$Z_{c1}=20\, \Omega$, $Z_{c2}=10\, \Omega$
}
\end{exercise}



\section{Solution des exercices}

\paragraph{Exercice~\ref{ex:mg-1}}~\\%

La méthode des noeuds consiste à écrire \ $n-1$ PLK \ en \
$n-1$ noeuds \ du circuit, un des noeuds étant choisi comme référence
des tensions. On exprime donc ces relations en fonction des potentiels
de noeuds \ ${\bf V}_N$~:
\[ {\bf I}_{sN} = {\bf G}_{N}\, {\bf V}_{N} \]
avec
\begin{enumerate}
	\item ${\bf G}_{N}$~: la matrice des conductances aux noeuds;
	\item ${\bf I}_{sN}$~: le vecteur des courants injectés aux noeuds
	par les sources indépendantes de courant.
\end{enumerate}
Le calcul de l'état électrique complet comporte ainsi les étapes suivantes.

\noindent{\bf 1. Choisir un noeud de référence}

On choisit le noeud e.

\noindent{\bf 2. Détermination de la matrice des conductances aux noeuds} \ ${\bf G}_N$ 

Cette matrice est donnée par
\[ {\bf G}_N = {\bf A} \, {\bf G}_B \, {\bf A}^T \]
avec \ ${\bf A}$ \ la matrice d'incidence réduite du graphe orienté du
circuit passifié représenté ci-dessous
\begin{center}
	\includegraphics[width=0.5\linewidth]{sol_exercices/ex1-18-1}
\end{center}
\[ {\bf A} \: = \: \left( \begin{array}{ccccccccc}
-1 & 0 & 0 & 0 & 1 & 0 & 0 & 1 & 0\\
0 & -1 & 0 & 0 & -1 & 1 & 0 & 0 & 1\\
0 & 0 & -1 & 0 & 0 & -1 & 1 & -1 & 0\\
0 & 0 & 0 & -1 & 0 & 0 & -1 & 0 & -1
\end{array} \right) \]
${\bf G}_B$ \ est la matrice des conductances de branches:
\[{\bf G}_B = \mbox{diag} \, (2,5,4,5,3,2,10,3,7)\]
Le circuit ne comporte que des résistances linéaires et des sources indépendantes de courant.
${\bf G}_N$ \ peut être déterminée directement par la règle d'inspection~:
\begin{enumerate}
	\item \'el\'ement diagonal \ $G_{ii} =$ somme des conductances des
	branches incidentes au noeud \ $i$~;
	\item élément non idagonal \ $ij =$ opposé de la conductance de la
	branche liant les noeuds \ $i$ et $j$~.
\end{enumerate}
\[ {\bf G}_N \: = \: \left( \begin{array}{cccc}
2+3+3 & -3 & -3 & 0\\
-3 & 3+5+2+7 & -2 & -7\\
-3 & -2 & 2+4+10+3 & -10\\
0 & -7 & -10 & 10+7+5
\end{array} \right)~\text{S}~. \]
On remarque que \ ${\bf G}_N$ \ est symétrique puisque le circuit ne
comporte que des conductances linéaires et des sources indépendantes
d'énergie.

\noindent{\bf 3. Détermination du vecteur des courants de noeuds} \ ${\bf I}_{sN}$

L'élément relatif au noeud \ $i$ \ est la somme des courants injectés
à ce noeud par les sources indépendantes.
\[ {\bf I}_{sN} \: = \: \left( \begin{array}{c}
-3 \\ 2.5 \\ 1+3 \\ 1
\end{array} \right)  \]

\noindent{\bf 4. Calcul des potentiels de noeuds}

De la relation \ ${\bf I}_{sN} = {\bf G}_N {\bf V}_N$~, on déduit:
\[ {\bf V}_N \: = \: {\bf G}_N ^{-1} {\bf I}_{sN}\: = \: 
\left( \begin{array}{c} V_a \\ V_b \\ V_c \\ V_d \end{array} \right)
\: = \: \left( \begin{array}{c} -0.112 \\ 0.306 \\ 0.395 \\ 0.322
\end{array} \right) \, \text{V~.} \] 
$V_a\, , \, V_b\, , \,V_c\, , \,V_d$ \
sont les potentiels des différents noeuds du circuit par rapport au
noeud de référence e.

\noindent{\bf 5. Tensions et courants de branches}

Les tensions de branches se déduisent des SLK
\[ {\bf U}_B = {\bf A}^T\, {\bf V}_N~. \]
Les courants de branches se déduisent de la loi d'Ohm écrite pour chaque branche
\[ {\bf I}_B = {\bf G}_B\, {\bf U}_B~. \]

\begin{tabular}{lll}
	\multicolumn{2}{l}{\underline{Tensions de branches}} & \underline{Courants de branches}\\[2mm]
	$\bullet$ branche 1 & & \\
	\parbox[c]{4cm}{\includegraphics[width=0.7\linewidth]{sol_exercices/ex1-18-4}} 
	& $\begin{array}{rcl}
	U_{1} &=& -V_a \\ &=& 0.112\, V \end{array}$
	& $\begin{array}{rcl}
	I_{1} &=& 2U_{1} \\ &=& 0.224\, A \end{array}$ \\[14mm]
	$\bullet$ branche 2 & & \\
	\parbox[c]{4cm}{\includegraphics[width=0.7\linewidth]{sol_exercices/ex1-18-2}}
	& $\begin{array}{rcl}
	U_{2} &=& -V_b \\ &=& -0.306\, V \end{array}$
	& $\begin{array}{rcl}
	I_{2} &=& 5U_{2} \\ &=& -1.53\, A \end{array}$ \\[20mm]
	$\bullet$ branche 3 & & \\
	\parbox[c]{4cm}{\includegraphics[width=0.7\linewidth]{sol_exercices/ex1-18-3}}
	& $\begin{array}{rcl}
	U_{3} &=& -V_c \\ &=& -0.395\, V \end{array}$
	& $\begin{array}{rcl}
	I_{3} &=& 4U_{3} \\ &=& -1.58\, A \end{array}$ \\[20mm]
	$\bullet$ branche 4 & & \\
	\parbox[c]{4cm}{\includegraphics[width=0.7\linewidth]{sol_exercices/ex1-18-5}}
	& $\begin{array}{rcl}
	U_{4} &=& -V_d \\ &=& -0.322\, V \end{array}$
	& $\begin{array}{rcl}
	I_{4} &=& 5U_{4} \\ &=& -1.61\, A \end{array}$ \\[20mm]
	$\bullet$ branche 5 & & \\
	\parbox[c]{4cm}{\includegraphics[width=0.6\linewidth]{sol_exercices/ex1-18-6}}
	& $\begin{array}{rcl}
	U_{5} &=& V_a-V_b\\ &=& -0.418\, V \end{array}$
	& $\begin{array}{rcl}
	I_{5} &=& 3U_{5} \\ &=& -1.254\, A \end{array}$ \\[14mm]
	$\bullet$ branche 6 & & \\
	\parbox[c]{4cm}{\includegraphics[width=0.6\linewidth]{sol_exercices/ex1-18-7}}
	& $\begin{array}{rcl}
	U_{6} &=& V_b-V_c \\ &=& -0.089\, V \end{array}$
	& $\begin{array}{rcl}
	I_{6} &=& 2U_{6} \\ &=& -0.178\, A \end{array}$ 
\end{tabular}

\begin{tabular}{lll}
	$\bullet$ branche 7 & & \\
	\parbox[c]{4cm}{\includegraphics[width=0.6\linewidth]{sol_exercices/ex1-18-8}}
	& $\begin{array}{rcl}
	U_{7} &=& V_c-V_d \\ &=& 0.073\, V \end{array}$
	& $\begin{array}{rcl}
	I_{7} &=& 10U_{7} \\ &=& 0.73\, A \end{array}$ \\[20mm]
	$\bullet$ branche 8 & & \\
	\parbox[c]{4cm}{\includegraphics[width=0.7\linewidth]{sol_exercices/ex1-18-9}}
	& $\begin{array}{rcl}
	U_{8} &=& V_a-V_c \\ &=& -0.507\, V \end{array}$
	& $\begin{array}{rcl}
	I_{8} &=& 3U_{8} \\ &=& -1.521\, A \end{array}$ \\[20mm]
	$\bullet$ branche 9 & & \\
	\parbox[c]{4cm}{ \includegraphics[width=0.7\linewidth]{sol_exercices/ex1-18-10}}
	& $\begin{array}{rcl}
	U_{9} &=& V_b-V_d \\ &=& -0.016\, V \end{array}$
	& $\begin{array}{rcl}
	I_{9} &=& 7U_{9} \\ &=& -0.112\, A \end{array}$
\end{tabular}

\paragraph{Exercice~\ref{ex:mg-2}}~\\%
Remarquons que ce circuit est équivalent à celui de l'exercice 1.18 si
toutes les branches sources de tension sont remplacées par des
branches sources de courant. 

La méthode des mailles consiste à écrire $b-(n-1)$ SLK pour un
ensemble de mailles fondamentales et à exprimer ces équations en
fonction des courants de mailles ${\bf I}_M$. Ces relations s'écrivent
:
\[{\bf V}_{sM}={\bf R}_M \, {\bf I}_M\]
avec :
\begin{enumerate}
	\item ${\bf R}_M$ : la matrice des résistances de mailles
	\item ${\bf V}_{sM}$ : le vecteur des f.e.m. de mailles imposées par les
	sources indépendantes de tension.
\end{enumerate}

Le calcul de l'état électrique complet comporte ainsi les étapes
suivantes.

\noindent{\bf 1. Transformer les branches contenant une source indépendante de courant.}

Par simple équivalence de source de courant - source de tension, on
transforme les branches de et ac comme indiqué ci-dessous :
\begin{center}
	\includegraphics[width=0.6\linewidth]{sol_exercices/ex1-19-1}
\end{center}
Le circuit complet se transforme en : 
\begin{center}
	\includegraphics[width=0.6\linewidth]{sol_exercices/ex1-19-2}
\end{center}

\noindent{\bf 2. Choisir un arbre et définir les mailles fondamentales
	correspondantes.}

Le graphe du circuit est représenté par :
\begin{center}
	\includegraphics[width=0.5\linewidth]{sol_exercices/ex1-19-3}
\end{center}
Le graphe orienté du circuit comporte 5 noeuds et 9 branches. Il y a
$b-(n-1)=9-(5-1)=5$ mailles fondamentales. On choisit par exemple
l'arbre ``3-5-6-7'' représenté en traits discontinus sur la figure ci-dessous. Les
maillons sont les branches 1,2,4,8 et 9. Chaque maille fondamentale,
constituée d'un maillon et de branches de l'arbre, est 
orientée selon le sens de référence adopté dans le maillon.
\begin{center}
	\includegraphics[width=0.7\linewidth]{sol_exercices/ex1-19-4}
\end{center}


\noindent{\bf 3.  Détermination de la matrice des résistances de mailles.}

Cette matrice est donnée par
\[{\bf R}_M={\bf B}{\bf R}_B{\bf B}^T\]
avec ${\bf B}$ la matrice des mailles fondamentales qui s'écrit :
\[{\bf B}=
\begin{pmatrix}
1 & 0 & -1 & 0 & 1 & 1 & 0 & 0 & 0\\
0 & 1 & -1 & 0 & 0 & 1 & 0 & 0 & 0\\
0 & 0 & -1 & 1 & 0 & 0 & -1 & 0 & 0 \\
0 & 0 & 0 & 0 & 0  & -1 & -1 & 0 & 1 \\
0 & 0 & 0 & 0 & -1 & -1 & 0 & 1 & 0
\end{pmatrix}\]
${\bf R}_B$ est la matrice des résistances de branches :
\[ {\bf R}_B=\mbox{diag}(\dfrac{1}{2},\dfrac{1}{5},\dfrac{1}{4},
\dfrac{1}{5},\dfrac{1}{3},\dfrac{1}{2},\dfrac{1}{10},
\dfrac{1}{3},\dfrac{1}{7})\]
Le circuit ne comporte que des résistances linéaires et des sources
indépendantes. ${\bf R}_M$ peut  être directement déterminée par la règle
d'inspection :
\begin{enumerate}
	\item élément diagonal $R_{ii}$ = somme des résistances
	des branches de la maille $i$;
	\item élément non-diagonal $R_{ij}$ = somme des résistances des branches communes aux
	mailles $i$ et $j$, prises avec le signe + si les sens de parcours des
	deux mailles coïncident, avec le signe - dans le cas contraire .
\end{enumerate}
\begin{eqnarray*}
	{\bf R}_M & = &
	\begin{pmatrix}
		\frac{1}{2}+\frac{1}{3}+\frac{1}{2}+\frac{1}{4} & 
		\frac{1}{2}+\frac{1}{4} & \frac{1}{4} & -\frac{1}{2} & -\frac{1}{3}-\frac{1}{2}\\
		\frac{1}{2}+\frac{1}{4}& \frac{1}{2}+\frac{1}{4}+\frac{1}{5} & 
		\frac{1}{4} & -\frac{1}{2} & -\frac{1}{2}\\
		\frac{1}{4} & \frac{1}{4}& \frac{1}{5}+\frac{1}{10}+\frac{1}{4} & 
		\frac{1}{10} & 0 \\
		-\frac{1}{2}& -\frac{1}{2}& \frac{1}{10}& 
		\frac{1}{7}+\frac{1}{10}+\frac{1}{2} & \frac{1}{2}\\
		-\frac{1}{3}-\frac{1}{2} & -\frac{1}{2} & 0 & \frac{1}{2}  & 
		\frac{1}{3}+\frac{1}{3}+\frac{1}{2}
	\end{pmatrix}\\
	& = & 
	\begin{pmatrix}
		1.5833 & 0.75 & 0.25 & _0.5 & -0.8333\\
		0.75 & 0.05 & 0.25 & -0.5 & -0.5 \\
		0.25 & 0.25 & 0.55 & 0.1 & 0  \\
		-0.5 & -0.5 & 0.1 & 0.743 & 0.5 \\
		-0.8333 & -0.5 & 0 & 0.5 & 1.1667
	\end{pmatrix}\,\, \Omega\text{~.}
\end{eqnarray*}
Remarquons que cette matrice est symétrique puisque le circuit ne
comporte que des résistances linéaires et des sources indépendantes
d'énergie.

\noindent{\bf 4. Détermination du vecteur des f.e.m. de mailles} 

L'élément
relatif à la maille $i$ est donné par la somme des f.e.m.  imposées
dans cette maille par les sources indépendantes comptées positivement si le
sens de la d.d.p. de la source coïncide avec le sens de parcours de la
maille  et négativement sinon.
\[{\bf V}_{sM}=
\begin{pmatrix}
-\frac{1}{4}\\
-\frac{1}{4}+\frac{1}{2}\\
\frac{1}{5}-\frac{1}{4}\\
0\\
1
\end{pmatrix} =
\begin{pmatrix}
-0.25 \\ 0.25 \\ -0.05 \\ 0 \\ 1
\end{pmatrix}\text{~V~.}
\]
\noindent{\bf 5. Calcul des courants de mailles}

Les courants de mailles sont les courants dans les maillons
1,2,4,8,9.


De la relation ${\bf V}_{sM}={\bf R}_M\, {\bf I}_M$, on déduit :
\[{\bf I}_M=\begin{pmatrix}
I_1\\ I_2\\I_4\\I_9\\I_8
\end{pmatrix}= {\bf R}_M^{-1}\, {\bf V}_{sM}=
\begin{pmatrix}
0.2239\\ 0.9675\\-0.6122\\-0.1114\\1.4794
\end{pmatrix}\text{~A~.}\]

\noindent{\bf 6. Courants de branches}

Les courants dans les branches restantes (les branches de l'arbre) se
déduisent des PLK \\${\bf I}_B={\bf B}^T\, {\bf I}_M$.
\begin{align*}
I_3 & = -I_4-I_2-I_1=-0.5792 \text{~A}\\
I_5 & = I_1-I_8=-1.2555 \text{~A}\\
I_6 & = I_1+I_2-I_8-I_9=-0.1766 \text{~A}\\
I_7 & = -I_4-I_9= 0.7236 \text{~A}
\end{align*}

{\bf 7. Tensions de branches}

Elles se déduisent de la loi d'Ohm et des lois de Kirchhoff.

\begin{tabular}{l@{\hspace{30mm}}l}
	\underline{Courants de branches} & \underline{Tensions de branches}\\[2mm]
	$\bullet$ branche 1 : & \\
	\parbox[c]{4cm}{
		\hfill \includegraphics[width=0.7\linewidth]{sol_exercices/ex1-19-5}} & 
	$\begin{array}{rcl}
	U_1 & = & \dfrac{0.2239}{2}\\\
	& =& 0.112\text{~V}
	\end{array}$ \\[14mm]
	$\bullet$  branche 2 : & \\
	\parbox[c]{4cm}{
		\hfill \includegraphics[width=0.7\linewidth]{sol_exercices/ex1-19-6}} & 
	$\begin{array}{rcl}
	U_2& =& 0.2\, . 0.9675  \\
	& = &  0.1935\text{~V}
	\end{array}$ \\[14mm]
	$\bullet$ branche 3 :& \\
	\parbox[c]{4cm}{
		\hfill \includegraphics[width=0.7\linewidth]{sol_exercices/ex1-19-7}}& 
	$\begin{array}{rcl}
	U_3& = & 0.25\, . (-0.5792) \\
	& =&  -0.1448\text{~V}
	\end{array}$ \\[14mm]
	$\bullet$ branche 4 :& \\
	\parbox[c]{4cm}{
		\hfill \includegraphics[width=0.7\linewidth]{sol_exercices/ex1-19-8}}& 
	$\begin{array}{rcl}
	U_4 & =& -0.2\, . (1+0.6122) \\
	& =&  -0.3224\text{~V}
	\end{array}$ \\[14mm]
	$\bullet$ branche 5 : & \\
	\parbox[c]{4cm}{
		\hfill \includegraphics[width=0.6\linewidth]{sol_exercices/ex1-19-9}}&
	$\begin{array}{rcl}
	U_5& = & -0.3333\, . 1.2555 \\
	& = &  -0.4185\text{~V}
	\end{array}$\\[14mm]
	
\end{tabular}

\begin{tabular}{l@{\hspace{30mm}}l}
	$\bullet$ branche 6 : & \\
	\parbox[c]{4cm}{
		\hfill \includegraphics[width=0.6\linewidth]{sol_exercices/ex1-19-10}} & 
	$\begin{array}{rcl}
	U_6 & =& -0.5\, . 0.1766 \\
	&=& -0.0883\text{~V}
	\end{array}$\\[14mm]
	$\bullet$ branche 7 :& \\
	\parbox[c]{4cm}{\hfill \includegraphics[width=0.6\linewidth]{sol_exercices/ex1-19-11}}&
	$\begin{array}{rcl}
	U_7 & = &0.1\, . 0.7236 \\
	& =&  0.0724\text{~V}
	\end{array}$\\[14mm]
	$\bullet$ branche 8 : & \\
	\parbox[c]{4cm}{\hfill \includegraphics[width=0.7\linewidth]{sol_exercices/ex1-19-12}}&
	$\begin{array}{rcl}
	U_8 &= &0.333\, . (1.4794-3) \\
	& = & -0.5069\text{~V}
	\end{array}$\\[14mm]
	$\bullet$ branche 9 :& \\
	\parbox[c]{4cm}{\hfill \includegraphics[width=0.6\linewidth]{sol_exercices/ex1-19-13}} & 
	$\begin{array}{rcl}
	U_9 & = &\dfrac{-0.1114}{7}\\
	& =&  -0.0159\text{~V}
	\end{array}$\\[14mm]
\end{tabular}

\paragraph{Exercice~\ref{ex:mg-3}}~\\%
{\bf 1. Réduction de la matrice de résistances de mailles } \ ${\bf R}_M$ \ - 
{\bf Elimination d'accès}

Selon la méthode des mailles, un circuit est vu comme un circuit
passifié à \ $M$ \ accès auquel sont connectées les sources
indépendantes de tension : 
\parbox[c]{5cm}{
\begin{center}
\includegraphics[width=0.9\linewidth]{sol_exercices/ex1-20-1}
\end{center}}
\parbox[c]{5cm}{On peut écrire
\[{\bf U} = {\bf R}_M \, {\bf I}\]
ou
\[{\bf V}_{sM} = {\bf R}_M \, {\bf I}\]}

Le circuit passifié est caractérisé par la matrice de résistances des mailles \ ${\bf R}_M$~.

Il y a un accès dans chaque maillon.

Si certains accès sont dépourvus de source indépendante : 
\begin{center}
\includegraphics[width=0.45\linewidth]{sol_exercices/ex1-20-2}
\end{center}
 on a, pour ces accès, \ $U = 0$~(accès $M$
 ici).

Si l'on élimine l'accès auquel aucune source n'agit, c'est-à-dire si
l'on élimine la variable \ $I_M$~, le circuit sera caractérisé par une
matrice de résistances de mailles réduite \ ${\bf
R}_{\mbox{red}}$~. Cette matrice s'obtient de la manière suivante~:

\[ \left( \begin{array}{c}
{\bf V}_{SM_1}\\ 0
\end{array} \right) \: = \: 
\left( \begin{array}{c|c}
{\bf R}_{M_{11}} & {\bf R}_{M_{12}}\\\hline
{\bf R}_{M_{21}} & R_{M_{22}}
\end{array} \right) \,
\left( \begin{array}{c}
{\bf I}_{M_1}\\
I_{M_2}
\end{array} \right) 
\: \begin{array}{c}
(1) \\ (2)
\end{array} \]

La matrice \ ${\bf R}_M$ \ est partitionnée selon quatre sous-matrices. Les
accès conservés sont repérés par l'indice 1, l'accès $M$ est l'accès
éliminé repéré par l'indice 2. 

La dernière relation (2) fournit
\[ I_{M_2} \: = -\, R_{M_{22}}^{-1} \, {\bf R}_{M_{21}} \, {\bf I}_{M_1}~~~~~(3) \]
Remplaçant dans (1), on trouve~:
\[ {\bf V}_{sM_1} \: = \left( {\bf R}_{M_{11}} - {\bf R}_{M_{12}} \, 
R_{M_{22}}^{-1}\, {\bf R}_{M_{21}} \right) {\bf I}_{M1}~, \]
relation cherchée et
\[ {\bf R}_{\mbox{red}} \: = \ {\bf R}_{M_{11}} - {\bf R}_{M_{12}} \, 
R_{M_{22}}^{-1} \, {\bf R}_{M_{21}}~. \] 
On a considéré ici
l'élimination d'un seul accès mais la procédure peut être généralisée
à plusieurs accès.

{\bf 2. Mise en équation du circuit via la méthode des mailles}

{\em A.  Graphe et choix de l'arbre}

L'arbre est choisi de façon à ce que les maillons correspondent aux
accès intéressants, en particulier les accès relatifs aux branches 1
et 2 où se trouvent les sources indépendantes d'énergie :
\begin{center}
\includegraphics[width=0.5\linewidth]{sol_exercices/ex1-20-3}
\end{center}



{\em B. Matrice de résistances de mailles}

De la règle d'inspection, on déduit la matrice de résistances des mailles~:

\[ {\bf R}_M \: = \: \left( \begin{array}{cc|c}
8 & 1 & 4\\
1 & 5 & -4\\ \hline
4 & -4 & 10
\end{array} \right) \,\, \Omega~.\]

Le vecteur des f.e.m. de mailles est donné par~:

\[ {\bf V}_{sM} \: = \left( \begin{array}{c}
6 \\ 10 \\ 0
\end{array} \right) \,\, \text{V~.}\]

\noindent{\bf 3. Elimination de l'accès relatif à la maille III}

Il n'y a pas de source indépendante agissant dans la maille III. Son
élimination conduit à la matrice réduite

\begin{eqnarray*}
{\bf R}_{\mbox{red}} &=& 
\left( \begin{array}{ccc} 8 && 1\\ 1 && 5 \end{array} \right) 
- \left( \begin{array}{c} 4 \\ -4 \end{array} \right) \: \frac{1}{10} \left( 4~~ -4\right)\\
&=& \left( \begin{array}{ccc} 6.4 && 2.6\\ 2.6 && 3.4 \end{array} \right) \Omega
\end{eqnarray*}

On a donc, vu des deux accès 11' et 22', la relation suivante~:

\[ \left( \begin{array}{c} 6\\ 10  \end{array} \right) 
\: = \: {\bf R}_{\mbox{red}} \left( \begin{array}{c} I_1\\ I_2 \end{array} \right)  \]

Le schéma équivalent correspondant est représenté par 
\begin{center}
\includegraphics[width=0.45\linewidth]{sol_exercices/ex1-20-4}
\end{center}

On déduit
\[ \left( \begin{array}{c} I_1\\ I_2 \end{array} \right) 
\: = \: {\bf R}_{\mbox{red}}^{-1} \left( \begin{array}{c} 6\\ 10  \end{array} \right) 
\: = \: \left( \begin{array}{c} -0.373\\ 3.227  \end{array} \right)  \text{A~.} \]

Le courant de la maille III, \ $I_3$ \, peut être déterminé à partir de
\ $I_1,I_2$ \ via la relation (3).
On trouve~:
\begin{eqnarray*}
I_3 \: = \: -\, \frac{1}{10} \left( 4 ~~-4 \right) 
\left( \begin{array}{c} I_1\\ I_2 \end{array} \right) 
&=& -\, 0.4 \, I_1 + 1.4 \, I_2 \\
&= & 1.436 \, \text{A~.}
\end{eqnarray*}

La connaissance des courants \ $I_1,I_2,I_3$ \ permet de déduire
l'état électrique complet du réseau.

{\bf 4. Détermination de} \ ${\bf R}_{\mbox{red}}$ \ {\bf par expérimentation}

La matrice \ ${\bf R}_{\mbox{red}}$ \ ne peut jamais être déterminée
par inspection. Par contre, elle peut être déterminée par
expérimentation~:
\begin{enumerate}
\item élément diagonal~: \ $R_{ii} = \left. U_i\right|_{I_i=1\, ,\, I_j=0\, ,\, j\neq i}$
\item élément non-diagonal\: \ $R_{ij} = \left. U_i\right|_{I_j=1\, ,\, I_k=0\, ,\, k\neq j}$
\item les éléments $R_{11}$ et $R_{21}$ sont  déterminés selon les conditions
de la figure ci-dessous : $I_1=1\, ,\, I_2=0$~.

\parbox[c]{6cm}{
\begin{center}
	\includegraphics[width=\linewidth]{sol_exercices/ex1-20-5}
\end{center}}
\parbox[c]{4cm}
{\begin{align*}
	I'_1 &= 1 \, . \, \frac{6}{10} \: = \: 0.6\, \text{A}\\
	I'_2 &= 0.4\, \text{A}
	\end{align*}}

\begin{xalignat*}{2}
R_{11} & = \left. U_1\right|_{I_1=1\, ,\, I_2=0} & 
R_{21} &= \left. U_2\right|_{I_1=1\, ,\, I_2=0}\\
& = 1 + 4I'_1 +3 & & = \, 1+4I'_2 \\
& = 6.4\, \Omega & & =  \, 2.6\, \Omega
\end{xalignat*}
\item  les éléments $R_{12}$ et $R_{22}$ sont déterminés selon les conditions
de la figure ci-dessous : \\$I_1=0\, ,\, I_2=1$~.
\parbox[c]{8cm}{
\begin{center}
\includegraphics[width=0.7\linewidth]{sol_exercices/ex1-20-6}
\end{center}}
\parbox[c]{5cm}{
\begin{align*}
I"_1 &= 1 \, . \, \frac{6}{10} \: = \: 0.6\, \text{A}\\
I"_2 &= 0.4\, \text{A}\\
\end{align*}}

\begin{xalignat*}{2}
	R_{12} & = \left. U_1\right|_{I_2=1\, ,\, I_1=0} &
	R_{22} & = \left. U_2\right|_{I_2=1\, ,\, I_1=0}\\
	& = \: 1 + 4I"_2 & & =  1 + 4I"_1\\
	& = 2.6\, \Omega && = 3.4 \, \Omega 
\end{xalignat*}
\end{enumerate}
			
			
\paragraph{Exercice~\ref{ex:mg-4}}~\\%
{\bf 1. Schéma équivalent de Norton et méthode des noeuds}

Selon la méthode des noeuds, comme indiqué à la figure ci-dessous,
un circuit est équivalent à un circuit passifié vu de \ $N=n-1$ \
accès, représenté par la matrice de conductances aux noeuds \ ${\bf
	G}_N$~; à chaque accès agit une source indépendante de courant
représentant les courants injectés à cet accès par les sources
indépendantes de courant présentes dans le circuit.

\noindent\parbox[c]{6cm}{
\begin{center}
\includegraphics[width=\linewidth]{sol_exercices/ex1-21-1}
\end{center}}
\parbox[c]{4cm}{
\[{\bf I} \, = -{\bf I}_{sN} + {\bf G}_N {\bf U}~~~~~~ (1)\]
où \ ${\bf U} = {\bf V}_N$, les potentiels de noeuds}

Chaque accès est défini entre un noeud et le noeud de référence.

Lorsque l'on ne s'intéresse qu'à un nombre réduit d'accès \ $(< N)$~,
on peut éliminer les accès non intéressants.
Soient :
\begin{enumerate}
\item $a$~: les accès intéressants;
\item $b$~: les accès non intéressants.
\end{enumerate}

On a~: \ ${\bf I}_{b}={\bf 0}$ \ et en partitionnant (1), on écrit~:
\[ \begin{array}{rclc}
\left( \begin{array}{c} {\bf I}_a \\ {\bf 0} \end{array} \right)
& = &
- \left( \begin{array}{c} {\bf I}_{s_a} \\  {\bf I}_{s_{b}}\end{array} \right)
+ \left( \begin{array}{ccc} 
{\bf G}_{a,a} &&  {\bf G}_{a,b}
\end{array} \right)
\left( \begin{array}{c} {\bf U}_{a} \\  {\bf U}_{b}\end{array} \right) 
& \begin{array}{c} (2) \\ (3) \end{array}
\end{array} \]

De (3), on tire~:
\[ {\bf U}_{b} \: = \: {\bf G}^{-1}_{b,b} \, {\bf I}_{s_{b}} 
- {\bf G}^{-1}_{b,b} \, {\bf G}_{b,a} \, {\bf U}_a \]
et remplaçant dans (2), on obtient~:
\[ {\bf I}_a\: = \: 
\left( -\, {\bf I}_{s_a} + {\bf G}_{a,b} \, {\bf G}^{-1}_{b,b} \, 
{\bf I}_{s_{b}} \right)
+ \left( {\bf G}_{a,a} - {\bf G}_{a,b} \, {\bf G}^{-1}_{b,b} \, 
{\bf G}_{b,a} \right) {\bf U}_a~.\]
On déduit les paramètres du schéma équivalent de Norton vu des $a$ accès~:
\begin{eqnarray*}
{\bf I}_{No} &=& {\bf I}_{s_a} - {\bf G}_{a,b} \, {\bf G}^{-1}_{b,b} \, {\bf I}_{s_{b}} \\
{\bf G}_{No} &=& {\bf G}_{a,a} - {\bf G}_{a,b} \, {\bf G}^{-1}_{b,b} \, {\bf G}_{b,a} 
\end{eqnarray*}

\noindent{\bf 2. Application de la méthode des noeuds}

{\em A. Transformation de la branche contenant une source indépendante de tension}

\begin{center}
\includegraphics[width=0.5\linewidth]{sol_exercices/ex1-21-2}
\end{center}


{\em B. Choix du noeud de référence}

Le circuit comporte 3 noeuds~: les noeuds \ 1 $(=C)$~, 2 $(=A)$ \ et \  1' $(=B)$~.
Choisissons le noeud 1' comme noeud de référence. 

Le circuit comporte 2 accès~:
\begin{enumerate}
\item l'accès 11'
\item un deuxième accès entre le noeud  $A$  et le noeud  1'.
\end{enumerate}

{\em C. Détermination de la matrice \ ${\bf G}_N$}

Le circuit comporte une source commandée de type CVT.\ ${\bf G}_N$ \ ne
peut être entièrement déterminée par la règle d'inspection.

L'élément (1,2) de la matrice doit être déterminé par
expérimentation. En effet, le CVT agit au noeud 1 et contribue donc
uniquement à la PLK à ce noeud. Il impose un courant fonction de la
tension \ $U_1$\ qui, vu le choix du noeud de référence, représente le
potentiel du noeud 2, c'est-à-dire la tension à l'accès 2.

La règle d'inspection fournit les éléments \ (1,1)~, (2,2) \ et \ (2,1)~:
\[ {\bf G}_N \: = \: 
\left( \begin{array}{cc}
\dfrac{1}{R_2} + \dfrac{1}{R_3} & \text{x} \\
-\, \dfrac{1}{R_3} & \dfrac{1}{R_0} + \dfrac{1}{R_1} + \dfrac{1}{R_3}
\end{array} \right) \]
La règle d'expérimentation, illustrée à la figure ci-dessous, fournit~:
\[  G_{12} = \left. I_1\right|_{V_2=1\, , \, V_1=0} \]
\begin{center}
\includegraphics[width=0.9\linewidth]{sol_exercices/ex1-21-3}
\end{center}
\begin{eqnarray*}
U_1 &=& V_2 = 1 V\\
I_{R_2} &=& 0 \quad \mbox{(tension nulle aux bornes de $R_2$)}\\
I_{R_3} &=& \frac{1}{R_3}\\
I_1 &=& -\, {R_3} - g U_1 \: = \, - \, \frac{1}{R_3} - g\\
G_{12} &=& -\, \frac{1}{R_3} - g
\end{eqnarray*}
Finalement :
\[ {\bf G}_N \: = \, \left( \begin{array}{cc}
\dfrac{1}{R_2} + \dfrac{1}{R_3} & -\, \dfrac{1}{R_3} - g\\
-\, \dfrac{1}{R_3} & \dfrac{1}{R_0} + \dfrac{1}{R_1} + \dfrac{1}{R_3}
\end{array} \right) \, \text{S~.}\]


{\em D.  Vecteur des courants injectés aux noeuds}

\[ {\bf I}_{sN} \: = \, \left( \begin{array}{c} 0 \\ \frac{E}{R_0} \end{array} 
\right) \, \text{A~.}\]

Seules les sources {\em indépendantes} doivent être prises en compte!

\noindent{\bf 3. Elimination de l'accès 2}

On cherche le schéma équivalent de Norton vu de l'accès 11$^{'}$. On
élimine l'accès 2 et on dérive successivement~:

\[ \left( \begin{array}{c} 0 \\\hline  G_0 E \end{array} \right)
\: = \: 
\left( \begin{array}{ccc} G_2+G_3 & | & -g-G_3  \\\hline  
-G_3 & | &  G_0 + G_1 + G_3
\end{array} \right) \:
\left( \begin{array}{c} V_1 \\\hline  V_2 \end{array}\right)  \]
\begin{eqnarray*}
I_{N_0} &=& 0 - (-g-G_3) \, . \, \frac{1}{G_0+G_1+G_3} \, . \ G_0E\\
&=& \frac{g+G_3}{G_0+G_1+G_3} \, . \, G_0E \: = \: 0.019\, E \,\, \text{A~}\\
G_{N_0} &=& G_2 + G_3 - (-g-G_3) \, \frac{1}{G_0+G_1+G_3} \, (-G_3)\\
&=& G_2+ G_3 - \frac{(g+G_3)\, G_3}{G_0+G_1+G_2} \\
&=& \frac{G_2G_0 + G_2G_1 + G_2G_3 + G_3G_0 + G_3G_1 - gG_3}{G_0+G_1+G_3}\\
&=& 7.17\, 10^{-3}\, \text{S~.}
\end{eqnarray*}

\noindent{\bf 4. Détermination directe de} \ $I_{No}$ \ et \ $G_{No}$

Etant donné la présence de la source commandée, il n'est pas possible
de procéder par simples réductions successives du circuit. La partie à
gauche de la source commandée ne peut pas être réduite.

$I_{No}$ \ est donné par le courant parcourant l'accès court-circuité.
Le circuit court-circuité peut se mettre sous la forme~de la figure ci-dessous.
\begin{center}
\includegraphics[width=0.9\linewidth]{sol_exercices/ex1-21-4}
\end{center}


Par application de la règle du diviseur de courant, on trouve~:
\begin{eqnarray*}
I_3 &=& \frac{G_3}{G_0+G_1+G_3} \, G_0E\\
I_1 &=& \frac{G_1}{G_0+G_1+G_3} \, G_0E\\
\mbox{et dès lors} ~~U_1 &=& \frac{I_1}{G_1} \: = \: \frac{G_0E}{G_0+G_1+G_3}
\end{eqnarray*}
Finalement~:
\begin{eqnarray*}
I_{No} \:  =\:   I_{cc} &=& I_3 + gU_1\\
&=& \frac{G_3+g}{G_0+G_1+G_3} \, G_0E
\end{eqnarray*}

$G_{No}$ est la conductance équivalente vue de l'accès du circuit
passifié. Le circuit passifié est représenté à la figure ci-dessous.
\begin{center}
\includegraphics[width=0.6\linewidth]{sol_exercices/ex1-21-5}
\end{center}
On dérive : 
\begin{eqnarray*}
G_{No} &=& \frac{I_1}{V_1} \: = \: \left. I_1 \right|_{V_1=1 \text{~V}}\\
I_1 &=& I_2+ I_3 - gU_1 = G_2 + \frac{(G_0+G_1)\, G_3}{G_0+G_1+G_3} - gU_1\\
\text{avec}\quad U_1 & =& \frac{I_3}{G_0+G_1} \: = \: \frac{G_3}{G_0+G_1+G_3}
\end{eqnarray*}
Finalement~: 
\[ G_{No} \: = \: \frac{G_2G_0 + G_2G_1 + G_2G_3 + G_0G_3 + G_1G_3 - gG_3}{G_0+G_1+G_3}~. \]

\paragraph{Exercice~\ref{ex:mg-5}}~\\%
{\bf Procédure~:}
\begin{enumerate}
	\item mettre le circuit en équations via la méthode des noeuds. Il y a
	3 noeuds plus le noeud de référence et donc 3 accès;
	\item élimination de l'accès inintéressant 31' et dérivation du schéma équivalent de Norton.
\end{enumerate}

\noindent{\bf 1. Application de la méthode des noeuds}

Transformons tout d'abord la branche contenant la source de tension :
\begin{center}
	\includegraphics[width=0.5\linewidth]{sol_exercices/ex1-22-1}
\end{center}

La matrice \ ${\bf G}_N$~, dérivée par la règle d'inspection, s'écrit~:
\[ {\bf G}_N \: = \: \left( \begin{array}{ccc|c}
8 && -5 &-2\\
-5 && 11 & -4 \\\hline
-2 && -4 & 7
\end{array} \right)\, \text{S~.} \]
Le vecteur des courants de noeuds est donné par~:
\[ {\bf I}_{sN} \: = \, \left( \begin{array}{c} 0 \\ 0 \\\hline 10 \end{array} \right) \, \text{A~.}\]

\noindent{\bf 2. Élimination de l'accès 31'}

Partitionnant les matrices, on dérive directement
\[ {\bf I}_{No} \: = \: 0 - \left( \begin{array}{c} -2 \\ -4 \end{array} \right) 
\, . \dfrac{1}{7} \, . \, 10 
\: = \: \left( \begin{array}{c} \dfrac{20}{7} \\ \dfrac{40}{7} \end{array} \right) \, \text{A} \]
et
\begin{eqnarray*}
	{\bf G}_{No} &=& \left( \begin{array}{cc} 8 & -5\\ -5 & 11 \end{array}\right)
	- \left( \begin{array}{c} -2 \\ -4  \end{array} \right) \: \dfrac{1}{7} \left( -2~~-4 \right)\\
	&=& \left( \begin{array}{cc} 7.43 &-6.14\\ -6.14 & 8.71 \end{array}\right)  \, \text{S}~.
\end{eqnarray*}

\noindent{\bf 3. Schéma équivalent}

Le circuit peut être représenté par le schéma équivalent de la figure ci-dessous avec : 
\begin{xalignat*}{2}
	G_{12} &= 6.14\, \text{S} & J_1 &=\frac{20}{7} \, \text{A}\\
	G_{11} &= 7.43 - 6.14 \; = \: 1.29\, \text{S} & J_2 &= \frac{40}{7}\, \text{A}\\
	G_{22} &= 8.71 - 6.14 \: = \: 2.57\, \text{S}~.
\end{xalignat*}
\begin{center}
	\includegraphics[width=0.85\linewidth]{sol_exercices/ex1-22-2}
\end{center}

\noindent{\bf 4. Détermination de} \ ${\bf G}_{No}$ \ {\bf par transfiguration}

La détermination du schéma équivalent de Norton revient ici à éliminer le noeud 3.
On ne peut pas directement appliquer la méthode de la transfiguration
pour transformer le circuit car il y a une source indépendante
d'énergie qui agit au centre de l'étoile. On peut par contre
l'utiliser pour rechercher la matrice \ ${\bf G}_{No}$ \ car pour cela
il faut passifier le circuit.

Ainsi, l'étoile
\begin{center}
	\includegraphics[width=0.4\linewidth]{sol_exercices/ex1-22-3}
\end{center}
devient après transformation

\parbox[c]{5cm}{
\begin{center}
\includegraphics[width=0.6\linewidth]{sol_exercices/ex1-22-4}
\end{center}}
\parbox[c]{5cm}{
avec 
\begin{eqnarray*}
	G'_{12} &=& \frac{2.4}{1+2+4} \: = \: \frac{8}{7}\, \,\text{S}\\
	G'_{10} &=& \frac{2.1}{1+2+4} \: = \: \frac{2}{7}\, \,\text{S}\\
	G'_{20} &=& \frac{4.1}{1+2+4} \: = \: \frac{4}{7}\, \,\text{S~.}
\end{eqnarray*}}

Le circuit passifié se simplifie comme indiqué à la figure ci-dessous.
\begin{center}
	\includegraphics[width=0.5\linewidth]{sol_exercices/ex1-22-5}
\end{center}

On dérive : 
\[ {\bf G}_{No} \: = \: \left( \begin{array}{cc}
6+\dfrac{10}{7} & -5 - \dfrac{8}{7}\\
-5 - \dfrac{8}{7} & 7 + \dfrac{12}{7} \end{array} \right) \]

{\bf 5. Détermination de} \ ${\bf I}_{No}$ \ {\bf par des essais en court-circuit}

Les courants de Norton \ $J_1$ et $J_2$ \ correspondent aux courants
parcourant les accès lorsque ceux-ci sont {\em simultanément}
court-circuités.
\[
\begin{pmatrix}
I_1\\I_2
\end{pmatrix} = -
\begin{pmatrix}
J_1\\J_2
\end{pmatrix} + {\bf G}_{No}
\begin{pmatrix}
V_1\\V_2
\end{pmatrix}\]
\[ \mbox{et~~~}{\bf I}_{No} \: = \: \left( \begin{array}{c} J_1 \\ J_2 \end{array} \right) 
\: = \, -  \left( \begin{array}{c} I_1 \\ I_2 \end{array} \right)_{V_1=V_2=0} \] 
\begin{center}
	\includegraphics[width=0.7\linewidth]{sol_exercices/ex1-22-6}
\end{center}
Des courants définis à la figure ci-dessus, on déduit :
\[ \begin{array}{rclcl}
I_3 &=& \dfrac{-2}{2+1+4} \, . \, 10 \: = \, -\, \dfrac{20}{7}\, \text{A} & 
\mbox{et} & J_1 = -I_3 = \dfrac{20}{7}\, \text{A} \\
I_4 &=& \dfrac{4}{2+1+4} \, . \, 10 \: = \, -\, \dfrac{40}{7}\, \text{A} & 
\mbox{et} & J_2 = -I_4 = \dfrac{40}{7}\, \text{A} 
\end{array} \]
		
\paragraph{Exercice~\ref{ex:mg-6}}~\\%
{\bf 1. Choix de la méthode} \ 

On recherche  $R_{18}$ et $R_{13}$~.

{\em A. Méthode des mailles}

\[ b-(n-1) \, = \, 12-7+1 \, = \, 6 \]
Il faut ajouter une branche correspondant à l'accès considéré (13 ou 18).

Il faut donc construire~:
\begin{enumerate}
	\item pour la recherche de \ $R_{18}$~, une matrice \ ${\bf R}_M$ \ de
	dimension \ $6$x$6$ \ et réduire cette matrice à l'accès 18;
	\item pour la recherche de \ $R_{13}$~, une nouvelle matrice \ ${\bf
		R}_M$ \ de dimension \ $6$x$6$ \ et réduire cette matrice à l'accès
	13.
\end{enumerate}

Les deux matrices \ ${\bf R}_M$ \ sont différentes puisque le graphe
du circuit considéré est différent~:
\begin{enumerate}
	\item pour la recherche de \ $R_{18}$~, on ajoute une branche
	correspondant à l'accès entre les noeuds 1 et 8;
	\item pour la recherche de \ $R_{13}$~, on ajoute une branche
	correspondant à l'accès entre les noeuds 1 et 3.
\end{enumerate}

{\em B. Méthode des noeuds}
\[n-1=7\]
\begin{enumerate}
	\item Recherche de \ $R_{18}$~: choisir 8 comme noeud de référence (ou
	1) et réduire la matrice \ ${\bf G}_N$ (7x7)\ à l'accès formé par le
	noeud 1 et le noeud 8.
	\item Recherche de\  $R_{13}$~: 
	\begin{enumerate}
		\item si 1 est le noeud de référence, réduire \ ${\bf G}_N$ \ à
		l'accès formé par le noeud 1 et le noeud 3;
		\item si 8 est le noeud de référence, réduire \ ${\bf G}_N$ \ aux 2
		accès $ \left( \begin{array}{c} 1\\ 8 \end{array} \right)$ et $ \left(
		\begin{array}{c} 3\\ 8 \end{array} \right)$~.
		
		On a
		\[ \left( \begin{array}{c} I_1 \\ I_3 \end{array} \right)
		\: = \: 
		\left( \begin{array}{cc} 
		G_{11} & G_{13}\\ G_{31} & G_{33}
		\end{array} \right) \, 
		\left( \begin{array}{c} V_1\\ V_3 \end{array} \right) \]
		
		$I_1$ et $I_3$ \ sont les courants aux accès 1 et 3;\\
		$V_1$ et $V_3$ \ sont les potentiels des noeuds 1 et 3 par rapport au noeud de référence.
		
		\begin{center}
			\includegraphics[width=\linewidth]{sol_exercices/ex1-23-1}
		\end{center}
		
		
		On a
		\[ R_{13} \, = \, \dfrac{V_1-V_3}{I} ~~~\mbox{avec}~~~ I=I_1=-I_3 \]
		Soit \ $I=1$~A, alors \ $R_{13}$ \ est donné par
		\[ R_{13} \, = \, V_1-V_3\]
		avec \ $V_1\, , \, V_3$ \ solutions de
		\[  \left( \begin{array}{c} V_1 \\ V_3 \end{array} \right) 
		\: = \: 
		\left( \begin{array}{cc} 
		G_{11} & G_{13}\\ G_{31} & G_{33}
		\end{array} \right) \, 
		\left( \begin{array}{c} 1\\ -1 \end{array} \right) \]
	\end{enumerate}
\end{enumerate}

{\bf 2. Méthode des noeuds}

On choisit la méthode des noeuds. Elle présente l'avantage de ne devoir
construire qu'une seule matrice \ ${\bf G}_N$ \ pour déterminer les
deux résistances demandées.

Nous allons réduire la matrice \ ${\bf G}_N$ \ par étapes successives
en éliminant à chaque fois un seul accès à la fois.

On montre que chaque étape est équivalente à une transfiguration
étoile \ $\leftrightarrow$ \ triangle qui élimine le noeud
correspondant à l'accès éliminé.

{\em A.  Matrice} \ ${\bf G}_N$ \ {\em complète}

\[ {\bf G}_N \: = \: 
\begin{array}{cc}
Noeud& 
\begin{array}{m{5mm}m{5mm}m{5mm}m{5mm}m{5mm}m{5mm}m{5mm}}
1 & 3 & 6 & 2 & 4 & 5 & 7
\end{array}\\
\begin{array}{r}
1\\ 3\\ 6\\ 2\\ 4\\ 5\\ 7
\end{array}
& \left(
\begin{array}{m{5mm}m{5mm}m{5mm}m{5mm}m{5mm}m{5mm}|m{5mm}}
3 & 0 & 0 & -1 & -1 & 0 & 0\\
0 & 3 & -1 & -1 & -1 & 0 & 0\\
0 & -1 & 3 & 0 & 0 & -1 & -1\\
-1 & -1 & 0 & 3 & 0 & -1 & 0\\
-1 & -1 & 0 & 0 & 3 & 0 & -1\\
0 & 0 & -1 & -1 & 0 & 3 & 0\\\hline
0 & 0 & -1 & 0 & -1 & 0 & 3
\end{array}\right)
\end{array}\]

{\em B. Elimination du noeud 7}

La matrice \ ${\bf G}_N$ \ réduite à 6 accès~s'écrit :
\[ \begin{array}{cc}
Noeud& \begin{array}{m{9mm}m{9mm}m{9mm}m{9mm}m{9mm}m{9mm}}
1 & 3 & 6 & 2 & 4 & 5 
\end{array}\\
\begin{array}{r}
1\\ 3\\ 6\\ 2\\ 4\\ 5
\end{array} 
& \left(\begin{array}{m{9mm}m{9mm}m{9mm}m{9mm}m{9mm}|m{9mm}}
3 & 0 & 0 & -1 & -1 & 0 \\
0 & 3 & -1 & -1 & -1 & 0 \\
0 & -1 & {\bf 2.67} & 0 & {\bf -0.33} & -1 \\
-1 & -1 & 0 & 3 & 0 & -1 \\
-1 & -1 & {\bf -0.33} & 0 & {\bf 2.67} & 0 \\\hline
0 & 0 & -1 & -1 & 0 & 3
\end{array}\right)
\end{array}\]

On remarque que seuls les éléments \ $G_{66}\,$, $G_{64}\,$, $G_{46}$
et $G_{44}$ sont modifiés.

Cette réduction correspond à la transformation étoile-triangle de la
 figure ci-dessous qui élimine le noeud 7.
\begin{center}
	\includegraphics[width=0.9\linewidth]{sol_exercices/ex1-23-2}
\end{center}

{\em C. Elimination du noeud 5}

Matrice \ ${\bf G}_N$ \ réduite à 5 accès~: seuls les éléments \
$G_{22}\,$, $G_{66}\,$, $G_{26}$ et $G_{62}$ \ sont modifiés.
\[ \begin{array}{cc}
Noeud& \begin{array}{m{9mm}m{9mm}m{9mm}m{9mm}m{9mm}}
1 & 3 & 6 & 2 & 4 
\end{array}\\
\begin{array}{r}
1\\ 3\\ 6\\ 2\\ 4
\end{array} 
& \left(\begin{array}{m{9mm}m{9mm}m{9mm}m{9mm}|m{9mm}}
3 & 0 & 0 & -1 & -1 \\
0 & 3 & -1 & -1 & -1  \\
0 & -1 & {\bf 2.33} & {\bf -0.33} & -0.33 \\
-1 & -1 & {\bf -0.33} & {\bf 2.67} & 0  \\\hline
-1 & -1 & -0.33  & 0 & 2.67
\end{array}\right)
\end{array}\]
Cette réduction correspond à la transformation étoile-triangle de la
 figure ci-dessous qui élimine le noeud 5.
\begin{center}
\includegraphics[width=0.9\linewidth]{sol_exercices/ex1-23-3}
\end{center}

{\em D. Elimination du noeud 4}

Matrice \ ${\bf G}_N$ \ réduite à 4 accès~: seuls les éléments \
$G_{11}\,$, $G_{66}\,$, $G_{33}\,$, $G_{13}\,$, $G_{16}\,$,
$G_{36}\,$, $G_{31}\,$, $G_{61}\,$ et $G_{63}$ \ sont modifiés.
\[ \begin{array}{cc}
Noeud& \begin{array}{m{9mm}m{9mm}m{9mm}m{9mm}}
1 & 3 & 6 & 2 
\end{array}\\
\begin{array}{r}
1\\ 3\\ 6\\ 2 
\end{array} 
& \left(\begin{array}{m{9mm}m{9mm}m{9mm}|m{9mm}}
{\bf 2.63} & {\bf -0.38} & {\bf -0.13} & -1  \\
{\bf -0.38} & {\bf 2.63}  & {\bf -1.13} & -1   \\
{\bf -0.13}  &  {\bf -1.13} & {\bf 2.29} &  -0.33 \\\hline
-1 & -1 & -0.33  & 2.67
\end{array}\right)
\end{array}\]

L'élimination du noeud 4 résulte d'une transformation étoile
\ $\leftrightarrow$ \ triangle généralisée à une étoile à 4 branches. Chaque \
$G_{ij}$ \ du circuit équivalent, liant les noeuds $i$ et $j$, est
donné par
\[ G_{ij} \: = \: \frac{G_i \, G_j}{\sum_i \, G_i} \]
Cette transformation est illustrée à la  figure ci-dessous.
\begin{center}
	\includegraphics[width=0.9\linewidth]{sol_exercices/ex1-23-4}
\end{center}

{\em E. Elimination du noeud 2}

Matrice \ ${\bf G}_N$ \ réduite à 3 accès~: tous les éléments sont modifiés.
\[ \begin{array}{cc}
Noeud& \begin{array}{m{9mm}m{9mm}m{9mm}}
1 & 3 & 6 
\end{array}\\
\begin{array}{r}
1\\ 3\\ 6
\end{array} 
& \left(\begin{array}{m{9mm}m{9mm}|m{9mm}}
{\bf 2.25} & {\bf -0.75} & {\bf -0.25}  \\
{\bf -0.75} & {\bf 2.25}  & {\bf -1.25}   \\\hline
{\bf -0.25}  &  {\bf -1.25} & {\bf 2.25}
\end{array}\right)
\end{array}\]
L'élimination est illustrée à la  figure ci-dessous.
\begin{center}
	\includegraphics[width=0.9\linewidth]{sol_exercices/ex1-23-5}
\end{center}

{\em F. Elimination du noeud 6}

On obtient la matrice \ ${\bf G}_N$ \ réduite aux 2 accès 18 et 38.
\[ {\bf G}_{\mbox{red}}=
\begin{array}{cc}
Noeud& \begin{array}{m{9mm}m{9mm}}
1 & 3 
\end{array}\\
\begin{array}{r}
1\\ 3
\end{array} 
& \left(\begin{array}{m{9mm}|m{9mm}}
2.22 & -0.89  \\\hline
-0.89  & 1.56
\end{array}\right)
\end{array}\]
L'élimination du noeud 6 est illustrée à la figure ci-dessous.
\begin{center}
	\includegraphics[width=0.9\linewidth]{sol_exercices/ex1-23-6}
\end{center}
On déduit~:
\[ \left( \begin{array}{c} V_1 \\ V_3 \end{array} \right) \, = \: 
{\bf G}_{\mbox{red}}^{-1} \left( \begin{array}{c} 1 \\ -1 \end{array} \right) \: \Rightarrow \: 
\left( \begin{array}{c} V_1 \\ V_3 \end{array} \right) \, = \, 
\left( \begin{array}{c} 0.25 \\ -0.5 \end{array} \right) \]
et \ $R_{13} \, = \, V_1 - V_3 \, = \, 0.75\, \Omega$~.


{\em G. Elimination du noeud 3}

Il reste la résistance équivalente vue de l'accès 18~:
\[ G_{18} \: = \: 2.22 - \dfrac{0.89^2}{1.56} \: = \: 1.71\, \text{S} \quad , \quad 
R_{18} = 0.58\, \Omega\]

\paragraph{Exercice~\ref{ex:mg-7}}~\\%

Nous allons tout d'abord déterminer le schéma équivalent de Thévenin du circuit vu du
détecteur \ $R_5$~. Il faut distinguer deux cas selon que l'on recherche $\frac{V_5}{E}$
ou $\frac{V_5}{I}$

{\bf A. Recherche de  $\frac{V_5}{E}$}

\begin{enumerate}
\item F.e.m. de Thévenin  $V_{{Th}_E}$

$V_{{Th}_E}$ \ est la tension apparaissant à vide à l'accès CD comme représenté ci-dessous : 
\begin{center} 
\includegraphics[width=0.3\linewidth]{sol_exercices/ex1-24-1}
\end{center} 

On dérive l'expression de $V_{{Th}_E}$ en fonction de $I$ :
\[\begin{array}{rcl}
I_1 &=& I\, . \, \dfrac{R_2+R_4}{R_1+R_2+R_3+R_4} \\
I_2 &=& I \, . \,\dfrac{R_1+R_3}{R_2+R_3+R_4+R_1} \\
V_{{Th}_E} &=& R_4I_2 - R_3I_1\\
&=& \dfrac{R_1R_4 -R_2R_3}{R_1+R_2+R_3+R_4} \, . \, I 
\end{array}\] 
Recherchons à présent l'expression de $V_{{Th}_E}$ \ en fonction de \ $E$. On a~:

\[ \begin{array}{c} 
(R_2+R_4)\, . \, I_2 + R_6I \: = \: E\\ 
\left(
\dfrac{R_2+R_4)(R_1+R_3)}{R_1+R_2+R_3+R_4} + R_6 \right) 
\, \dfrac{R_1+R_2+R_3+R_4}{R_1R_4
- R_2R_3} \, V_{Th} \: = \: E \\
V_{{Th}_E} \: = \: E \, . \, \dfrac{R_1R_4 -R_2R_3}{(R_2+R_4)(R_1+R_3) + 
\left( \sum^4_{i=1} R_i\right) R_6} 
\end{array} \]

Dans le cas particulier où \ $R_6=0$~, $V_{{Th}_E}$ se réduit à :
\[ V_{{Th}_E} \: = \: E \,. \, \dfrac{R_1R_4 - R_2R_3}{(R_2+R_4)(R_1+R_3)} \]

\item Résistance équivalente de Thévenin \ ${R_{Th}}_E$

Considérons le circuit passifié représenté ci-dessous : 
\begin{center} 
\includegraphics[width=0.5\linewidth]{sol_exercices/ex1-24-2}
\end{center} 

Nous allons rechercher sa résistance équivalente en appliquant la méthode des mailles et
réduisant la matrice \ ${\bf R}_M$ \ trouvée à l'accès CD.

Ajoutant la branche relative à l'accès, le graphe du circuit est représenté ci-dessous :
\begin{center} 
\includegraphics[width=0.5\linewidth]{sol_exercices/ex1-24-3}
\end{center} 

L'arbre est choisi de façon à laisser la branche de l'accès dans un maillon. La règle
d'inspection fournit la matrice de résistances de mailles \ ${\bf R}_M$~ : 
\[ {\bf R}_M =
\left( \begin{array}{ccc} R_1+R_2 & R_1+R_2 & R_1\\ R_1+R_2 & R_1+R_2+R_3+R_4 & R_1+R_3\\
R_1 & R_1+R_3 & R_1+R_3+R_6 \end{array}\right)\]

La résistance équivalente de Thévenin, résistance équivalente vue du premier accès
(branche $R_5$), est obtenue par réduction de ${\bf R}_M$ à cet accès.

Posons : 
\begin{gather*} 
{\bf R}_{aa}=\left(R_1+R_2\right)\\ 
{\bf R}_{ab}=\left(
\begin{array}{cc} R_1+R_2 & R_1 
\end{array} \right)\\ 
{\bf R}_{ba}=\left( 
\begin{array}{c}
R_1+R_2 \\ R_1
\end{array} \right)\\ 
{\bf R}_{bb}=\left( 
\begin{array}{cc} R_1+R_2+R_3+R_4
& R_1+R_3\\ R_1+R_3 & R_1+R_3+R_6 
\end{array} \right) 
\end{gather*}

On a : 
\[R_{Th_E}={\bf R}_{aa}-{\bf R}_{ab}{\bf R}_{bb}^{-1}{\bf R}_{ba}\] 
\[ {\bf R}_{bb}^{-1} = \frac{1}{\Delta}\left(
\begin{array}{cc} R_1+R_3+R_6 & -R_1-R_3\\
-R_1-R_3 & R_1+R_2+R_3+R_4
\end{array}\right) \] 
avec 
\begin{align*}
\Delta & =
(R_1+R_2+R_3+R_4)(R_1+R_3+R_6)-(R_1+R_3)^2\\
&=((R_1+R_3)+(R_2+R_4))((R_1+R_3)+R_6)-(R_1+R_3)^2\\ 
&=(R_1+R_3)(R_2+R_4)+R_6\sum_{i=1}^4R_i 
\end{align*}

\begin{align*}
{\bf R}_{bb}^{-1}{\bf R}_{ba}& = \frac{1}{\Delta}\left( 
\begin{array}{cc}
R_1+R_3+R_6 & -R_1-R_3\\ -R_1-R_3 & R_1+R_2+R_3+R_4 \end{array}\right) 
\left(
\begin{array}{c} R_1+R_2 \\ R_1 \end{array} \right) \\ 
&= \frac{1}{\Delta}\left(
\begin{array}{c} R_6(R_1+R_2)+R_2(R_1+R_3)\\ R_1R_4-R_2R_3 \end{array}\right) 
\end{align*}
\begin{align*} 
{\bf R}_{ab}\bf R_{bb}^{-1}{\bf R}_{ba}& = 
\frac{1}{\Delta}\left(
\begin{array}{cc} R_1+R_2 & R_1 \end{array}\right) 
\left( \begin{array}{c}
R_6(R_1+R_2)+R_2(R_1+R_3)\\ R_1R_4-R_2R_3 \end{array}\right) \\
& =
\frac{1}{\Delta}\left((R_1+R_2)^2R_6+R_1^2(R_2+R_4)+R_2^2(R_1+R_3)\right)\\ 
\end{align*}
\[ R_{Th_E} =
R_1+R_2-\frac{1}{\Delta}\left((R_1+R_2)^2R_6+R_1^2(R_2+R_4)+R_2^2(R_1+R_3)\right) =
\frac{N}{\Delta}\] 
avec 
\begin{align*} 
N &
=(R_1+R_2)(R_1+R_3)(R_2+R_4)+(R_1+R_2)^2R_6+(R_1+R_2)(R_3+R_4)R_6\\ 
& -(R_1+R_2)^2R_6-R_1^2(R_2+R_4)-R_2^2(R_1+R_3)\\ 
& =(R_1R_2+R_1R_3+R_2R_3)(R_2+R_4)-R_2^2R_1-R_2^2R_3\\ 
& + (R_1+R_2)(R_3+R_4)R_6\\ 
& =R_1R_2R_3+R_1R_2R_4+R_1R_3R_4+R_2R_3R_4+ (R_1+R_2)(R_3+R_4)R_6\\ 
& =R_1R_3(R_2+R_4)+R_2R_4(R_1+R_3)+(R_1+R_2)(R_3+R_4)R_6 
\end{align*} 
Finalement : 
\[R_{Th_E}= \frac{R_1R_3(R_2+R_4)+R_2R_4(R_1+R_3)+(R_1+R_2)(R_3+R_4)R_6}
{(R_1+R_3)(R_2+R_4)+R_6\sum_{i=1}^4R_i}\]

\item Expression de \ $\frac{V_5}{E}$

On connecte la résistance \ $R_5$ \ au schéma équivalent de Thévenin.

\parbox[c]{5cm}{
\begin{center} \includegraphics[width=0.9\linewidth]{sol_exercices/ex1-24-4} 
\end{center}} 
\parbox[c]{5cm}{ \[ V_5 \:
= \: R_5 \, \frac{V_{Th_E}}{R_5 + R_{Th_E}} \]}

L'expression générale de \ $V_5/E$ s'écrit : 
\[\frac{V_5}{E}=\frac{R_1R_4-R_2R_3}
{(R_1+R_3)(R_2+R_4)+R_6\sum_{i=1}^4R_i+\frac{1}{R_5}\sum_{i\neq j\neq k}^4R_iR_jR_k
+\frac{R_6}{R_5}(R_1+R_2)(R_3+R_4)}\]

Cas particuliers : 
\begin{enumerate}
\item $R_6=0$, $R_5=\infty$
\[\frac{V_5}{E}=\frac{R_1R_4-R_2R_3}{(R_1+R_3)(R_2+R_4)}\] 
\item $R_6=0$, $R_5\neq \infty$
\[\frac{V_5}{E}=\frac{R_1R_4-R_2R_3}{(R_1+R_3)(R_2+R_4)+\frac{1}{R_5}\sum_{i\neq j\neq
		k}^4R_iR_jR_k}\] 
\item $R_6\neq 0$, $R_5= \infty$
\[\frac{V_5}{E}=\frac{R_1R_4-R_2R_3}{(R_1+R_3)(R_2+R_4)+R_6\sum_{i=1}^4R_i}\]
\end{enumerate}
\end{enumerate}
{\bf B. Recherche de  $\frac{V_5}{I}$}

Contrairement à $E$, le courant $I$ est dépendant de la résistance du détecteur $R_5$. Le
schéma équivalent de Thévenin vu de l'accès CD doit être déterminé en supposant $I$ imposé. Pour cela, on applique le théorème de substitution : la branche $E$-$R_6$ est remplacée par une source de courant délivrant le courant $I$ comme indiqué ci-dessous :
\begin{center} 
\includegraphics[width=0.5\linewidth]{sol_exercices/ex1-24-5}
\end{center} 

\begin{enumerate}
\item  f.e.m. de Thévenin  $V_{{Th}_I}$

On a déterminé plus haut l'expression de la tension aux bornes du circuit à vide :
\[V_{{Th}_I}= \dfrac{R_1R_4 -R_2R_3}{R_1+R_2+R_3+R_4} \, . \, I \]

\item Résistance équivalente de Thévenin \ $R_{{Th}_I}$

On déduit directement  : 
\begin{center} 
	\includegraphics[width=0.55\linewidth]{sol_exercices/ex1-24-6}
\end{center} 
\[ R_{{Th}_I} \: = \: \frac{(R_1+R_2)(R_3+R_4)}{\sum^4_{i=1} R_i} \] 

\item  Expression de \ $\frac{V_5}{I}$

On a le schéma équivalent de Thévenin auquel on connecte la résistance \ $R_5$~.
\begin{eqnarray*} 
	I_{R_5} &=& \frac{V_{{Th}_I}}{R_5+R_{{Th}_I}} \\ 
	\mbox{et}~~V_5 &=& R_5
	I_{R_5} \; = \; R_5 \, \frac{V_{{Th}_I}}{R_5+R_{{Th}_I}} 
\end{eqnarray*}	

Remplaçant $V_{{Th}_I}$ et $R_{{Th}_I}$ par leur expression trouvée plus haut, le rapport \ $V_5/I$ \ s'écrit :
\[\frac{V_5}{I}=\frac{R_1R_4-R_2R_3}{\sum_{i=1}^4R_i+\frac{1}{R_5} (R_1+R_2)(R_3+R_4)}\]
Dans le cas particulier où $R_5=\infty$ :
\[\frac{V_5}{I}=\frac{R_1R_4-R_2R_3}{\sum_{i=1}^4R_i}\]
	
\end{enumerate}
\paragraph{Exercice~\ref{ex:mg-8}}~\\%
Nous allons résoudre cet exercice via deux approches~:
\begin{enumerate}
	\item par application de la méthode des mailles;
	\item par transfiguration étoile \ $\leftrightarrow$ \ triangle.
\end{enumerate}

{\bf 1. Application de la méthode des mailles}

Le graphe du circuit est donné ci-dessous.
La branche 1 est relative à l'accès. L'arbre est choisi de manière à
laisser cette branche dans un maillon.
\begin{center}
	\includegraphics[width=0.6\linewidth]{sol_exercices/ex1-25-1}
\end{center}

Les relations \ $U$-$I$ aux accès s'écrivent~:
\[ \left( \begin{array}{c} U_1 \\ U_2 \\ U_3 \\ U_4 \end{array} \right)
\: = \, -\, {\bf V}_{sM} + {\bf R}_M
\left(  \begin{array}{c} I_1 \\ I_2 \\ I_3 \\ I_4 \end{array} \right)~. \]

Seul l'accès 1 nous intéresse et \ $U_2 = U_3 = U_4 = 0$~. Le vecteur
des f.e.m. de mailles est donné par~:
\[ {\bf V}_{sM} \: = \: \left(  \begin{array}{c} 0 \\ 0 \\ 12 \\ 0 \end{array} \right)\, \text{V~.} \]

La règle d'inspection fournit la matrice \ ${\bf R}_M$~:
\[ {\bf R}_M \: = \: 
\left( \begin{array}{rrrr}
110 & 100 & 80 & -20\\
100 & 200 & 80 & -20\\
80 & 80 & 105 & 20\\
-20 & -20 & 20 & 50 
\end{array} \right)\, \Omega \text{~.} \]

On a~:
\[ \left( \begin{array}{c} U_1 \\\hline
0 \\ 0 \\ 0 \end{array} \right)
\: = \, -\, 
\left( \begin{array}{c} 0 \\\hline
0 \\ 12 \\ 0 \end{array} \right)
+ \left( \begin{array}{r|rrr} 
110 & 100 & 80 & -20 \\\hline
100 & 200 & 80 & -20\\
80 & 80 & 105 & 20\\
-20 & -20 & 20 & 50
\end{array} \right) \, 
\left(  \begin{array}{c} I_1 \\ I_2 \\ I_3 \\ I_4 \end{array} \right)~.  \]

La réduction à l'accès 1, c'est-à-dire l'élimination des variables \
$I_2\, ,\, I_3\, ,\, I_4$ \ fournit successivement~:
\[ \left(  \begin{array}{c} I_2 \\ I_3 \\ I_4 \end{array} \right)
\: = \, - \, 
\left( \begin{array}{rrr} 
200 & 80 & -20 \\
80 & 105 & 20\\
-20 & 20 & 50
\end{array} \right)^{-1}
\left( \left( \begin{array}{c} 0 \\ -12 \\ 0 \end{array} \right)
+ \left( \begin{array}{r} 100 \\ 80 \\ -20 \end{array} \right)
I_1 \right) \]
et
\begin{eqnarray*}
	U_1 &=&  110\, I_1 + \left( 100~~80~~-20 \right) \, 
	\left( \begin{array}{c} I_2 \\ I_3 \\ I_4 \end{array} \right)\\
	&=& -\left( 100~~80~~-20 \right) \, 
	\left( \begin{array}{rrr} 
		200 & 80 & -20\\
		80 & 105 & 20\\
		-20 & 20 & 50 
	\end{array} \right)^{-1}
	\left( \begin{array}{c} 0 \\ -12 \\ 0 \end{array} \right)\\
	&& + \left( 110 - \left( 100~~80~~-20 \right) \, 
	\left( \begin{array}{rrr}
		200 & 80 & -20\\
		80 & 105 & 20\\
		-20 & 20 & 50
	\end{array} \right) ^{-1}
	\left( \begin{array}{r} 100\\ 80\\ -20 \end{array} \right) 
	\right) \, I_1\\
	&=& V_{Th} + R_{Th} I_1\\
	V_{Th} &=& 9.706\, V\\
	R_{Th} &=& 20.846\, \Omega
\end{eqnarray*}

Pour déterminer la puissance fournie par \ $E=12V$ \ si l'accès 11'
est à vide, il faut déterminer \ $I_3$ \ lorsque \ $I_1=0$~.

On a, si \ $I_1=0$~:
\[ \left( \begin{array}{c} I_2\\ I_3\\ I_4 \end{array} \right) 
\, = \, -\left( \begin{array}{rrr}
200 & 80 & -20\\
80 & 105 & 20\\
-20 & 20 & 50
\end{array} \right) ^{-1}
\left( \begin{array}{c} 0\\ -12\\ 0 \end{array} \right) 
\; = \; 
\left( \begin{array}{c} 
-\, 9.705~10^{-2}\\ 0.2118 \\ -0.1236
\end{array} \right) \text{~A} \]
c'est-à-dire \ $I_3 = 0.2118$ A \ et

\parbox[c]{5cm}{\begin{center}
\includegraphics[width=0.4\linewidth]{sol_exercices/ex1-25-2}
\end{center}}
\parbox[c]{5cm}{
$p_E = 12\, I_3 = 2.54\, W$

$E~\mbox{~fournit~} ~2.54\, W$}

{\bf 2. Transfiguration}

On transforme l'étoile centrale comme suit : 
\begin{center}
\includegraphics[width=0.7\linewidth]{sol_exercices/ex1-25-3}
\end{center}

\begin{eqnarray*}
\frac{1}{R_{12}} &=& \dfrac{G_1G_2}{G_1+G_2+G_3}
\; = \; \dfrac{~\dfrac{1}{20}\, . \, \dfrac{1}{20}~}
{~\dfrac{1}{20} + \dfrac{1}{20} + \dfrac{1}{80}~}\: 
\Rightarrow \; R_{12} \: = \: 45\, \Omega\\
\dfrac{1}{R_{13}} &=& \dfrac{G_1G_3}{G_1+G_2+G_3}
\; = \; \dfrac{1}{180} \; \Rightarrow \; R_{13} \: = \: 180\, \Omega\\
\dfrac{1}{R_{23}} &=& \dfrac{G_2G_3}{G_1+G_2+G_3}
\; = \; \dfrac{1}{180} \; \Rightarrow \; R_{23} \: = \: 180\, \Omega
\end{eqnarray*}

Le circuit devient~:
\begin{center}
\includegraphics[width=\linewidth]{sol_exercices/ex1-25-4}
\end{center}

et finalement~:

\parbox[c]{6cm}{\begin{center}
\includegraphics[width=\linewidth]{sol_exercices/ex1-25-5}
\end{center}}
\parbox[c]{4cm}{
\begin{eqnarray*}
	V_{Th} &=& 64.29 \, \frac{11.68}{13.05 + 64.29} \: \\&= &\: 9.706\, V \\
	R_{Th} &=& 10 + 64.29\, // \, 13.05 \:\\& =& \: 20.846\, \Omega 
\end{eqnarray*}}

Pour calculer la puissance produite par  la source \ $E=12V$,
revenons au circuit où celle-ci apparaît,  comme illustré ci-dessous : 
\begin{center}
	\includegraphics[width=0.8\linewidth]{sol_exercices/ex1-25-6}
\end{center}

L'accès est à vide \ $\Rightarrow \: I_1 = 0$ \ et \ $U_1 = V_{Th} =
9.706\, V$~.

On dérive successivement~:
\begin{eqnarray*} 
	I_2 &=& \frac{U_1}{64.29} \\
	U_2 &=& U_1 + 8.18\, I_2 \\
	I_E &=& \frac{U_2 -12}{5} \: = \, -\, 0.2117\, A~.
\end{eqnarray*}

Puissance fournie par E~:
\[ p_E \: = \,-\, 12\, I_E \: = \: 2.54\, \text{W~.} \] 



\paragraph{Exercice~\ref{ex:mg-9}}~\\%
Il faut tout d'abord déterminer le schéma équivalent de Thévenin du
circuit vu de l'accès $AB$ :
\begin{center}
	\includegraphics[width=0.5\linewidth]{sol_exercices/ex1-26-1}
\end{center}

On utilise pour cela la méthode des mailles. On réduira ensuite le
système obtenu à l'accès AB.

Le graphe du circuit est donné ci-dessous. Le choix de l'arbre laisse l'accès (branche 3)
dans un maillon.
\begin{center}
	\includegraphics[width=0.5\linewidth]{sol_exercices/ex1-26-2}
\end{center}

Le vecteur des f.e.m. de mailles s'écrit~:
\[ {\bf V}_{sM} \: = \, 
\left( \begin{array}{c}
100\\ -100 \\ 100-200 \\ 100-200 
\end{array} \right)
\, = \, 
\left( \begin{array}{r}
100\\ -100 \\ -100 \\ -100 
\end{array} \right)\,\text{V~.} \]

La matrice \ ${\bf R}_M$ \ ne peut être entièrement déterminée par la
règle d'inspection suite à la présence de la source commandée de type
CVT. La SLK dans la maille IV est affectée par cette source; seuls les
éléments de la 4ème ligne de \ ${\bf R}_M$ \ doivent être déterminés
par expérimentation. De plus, le CVT introduit une tension qui dépend
du courant \ $I_{\Delta}$ \ c'est-à-dire de courant de la maille II et
finalement seul l'élément 42 de \ ${\bf R}_M$ \ doit être calculé par
expérimentation.

On a~:
\[R_{42}=U_4|_{I_2=1,I_1=I_3=I_4=0}\]
\begin{center}
	\includegraphics[width=0.5\linewidth]{sol_exercices/ex1-26-3}
\end{center}

\begin{eqnarray*}
	U_4 &=& 14\, I_{\Delta} - 3 \: = \: 14-3 \, = \, 11\\
	I_{\Delta} &=& I_2 \, = \, 1\\
	& \Rightarrow & R_{42} \, = \, 11\,\, \Omega\text{~.}
\end{eqnarray*}
On obtient~:
\[ {\bf R}_M \: = \: 
\left( \begin{array}{rrrr}
5 & -5 & 5 & 3\\
-5 & 25 & -5 & -3\\ 
5 & -5 & 10 & 7\\
3 & 11 & 7 & 7
\end{array} \right)\,\Omega\text{~.} \]
L'élimination des accès 2,3,4 fournit~:
\begin{eqnarray*}
	V_{Th} &=& -\, 100 - \left( -5 ~~5~~3 \right) \, 
	\left( \begin{array}{rrr}
		25 & -5 & -3\\
		-5 & 10 & 7\\
		11 & 7 & 7
	\end{array} \right) ^{-1}
	\left( \begin{array}{c} 100\\ 100\\ 100 \end{array} \right)\\
	&=& -\, 150\, \text{~V}\\
	R_{Th} &=& 5 - \left( -5 ~~5 ~~3 \right) \,
	\left( \begin{array}{rrr}
		25 & -5 & -3\\
		-5 & 10 & 7\\
		11 & 7 & 7
	\end{array} \right) ^{-1}
	\left( \begin{array}{r} -5\\ 5\\ 3 \end{array} \right)\\
	&=& 2.5\, \Omega \,\text{~.}
\end{eqnarray*}
On connecte \ $R_0$ \ à l'accès~:
\[I \: = \: \dfrac{V_{Th}}{R_{Th} + R_0}\]

La puissance dissipée par \ $R_0$ \ vaut~:
\[ P_{R_0} \: = \, \left( \frac{150}{R_{Th}+ R_0} \right)^2.\, R_0 \:= \: 1000\,\, \text{W~.} \]
On déduit l'équation définissant \ $R_0$~:
\begin{eqnarray*}
	(150)^2\, R_0 
	&=& 1000 \, (2.5 + R_0)^2\\
	&=& 10^3\, R_0^2 + 5000\, R_0 + 6250~.
\end{eqnarray*}
Cette équation fournit deux solutions pour \ $R_0$~:
\[ R_0 =
\begin{cases}
17.14 \,\,\Omega\\
0.365\,\,\Omega
\end{cases}\]


\paragraph{Exercice~\ref{ex:mg-10}}~\\%
{\bf Procédure} 
\begin{enumerate}
	\item mise en equations du circuit via la méthode des noeuds;
	\item réduction à l'accès 14 et dérivation du schéma équivalent de
	Norton vu de cet accès;
	\item insertion de la résistance $R$ à l'accès 14 et calcul de la
	puissance dissipée par cette résistance.
\end{enumerate}

\noindent{\bf 1. Application de la méthode des noeuds au circuit}

\begin{center}
	\includegraphics[width=0.5\linewidth]{sol_exercices/ex1-27-1}
\end{center}
Le circuit devra être ultérieurement réduit à l'accès 14, on peut
donc choisir soit le noeud 1 soit le noeud 4 comme noeud de
référence. Choisissons le noeud 4.

Vu la présence de sources commandées, seuls les éléments ci-dessous
de  la matrice de conductances aux
noeuds peuvent être  déterminés par la règle d'inspection. 
\[{\bf G}_N=
\begin{matrix}
1\\
2\\
3
\end{matrix}
\begin{pmatrix}
G_{12}+G_{13} & -G_{12} & -G_{13}\\
\times & \times & \times \\
\times & -G_{23} & G_{13}+G_{23}+G_{34}
\end{pmatrix}\]

Les autres éléments se déduisent 
\begin{enumerate}
	\item soit par expérimentation :
	\begin{enumerate}
		\item élément 21 : 
		
		\parbox{6cm}{\includegraphics[width=0.95\linewidth]{sol_exercices/ex1-27-2}}
		\parbox{4cm}{\begin{align*}
			{\bf G}_{N21}&=I_2|_{V_1=1;V_2=V_3=0}\\
			& = -G_{12}+g
			\end{align*}}
		\item  élément 22 :
		
		\parbox{6cm}{\includegraphics[width=0.95\linewidth]{sol_exercices/ex1-27-3}}
		\parbox{4cm}{\begin{align*}
			{\bf G}_{N22}&=I_2|_{V_2=1;V_1=V_3=0}\\
			& = G_{12}+G_{23}+G_{24}+g
			\end{align*}}
		\item  élément 23 :
		
		\parbox{6cm}{\includegraphics[width=0.95\linewidth]{sol_exercices/ex1-27-4}}
		\parbox{4cm}{\begin{align*}
			{\bf G}_{N23}&=I_2|_{V_3=1;V_1=V_2=0}\\
			& = -G_{23}-g
			\end{align*}}
		\item  élément 31 :
		
		\parbox{6cm}{\includegraphics[width=0.95\linewidth]{sol_exercices/ex1-27-5}}
		\parbox{4cm}{\begin{align*}
			{\bf G}_{N31}&=I_3|_{V_1=1;V_2=V_3=0}\\
			& = -G_{13}-g
			\end{align*}}
	\end{enumerate}
	\item soit en écrivant directement les PLK aux noeuds 2 et 3 :
	\begin{gather*}
	G_{12}(V_2-V_1)+G_{23}(V_2-V_3)+G_{24}V_2+gV_1-g(V_3-V_2)=-J\\
	G_{13}(V_3-V_1)+G_{23}(V_3-V_2)+G_{34}V_3-gV_1=0
	\end{gather*}
\end{enumerate}

Finalement, la matrice de conductances aux noeuds s'écrit :

\begin{eqnarray*}
	{\bf G}_N & = &\left(
	\begin{array}{c|cc}
		G_{12}+G_{13} & -G_{12} & -G_{13}\\ \hline
		-G_{12}+g & G_{12}+G_{23}+G_{24}+g & -G_{23}-g\\
		-G_{13}-g & -G_{23} & G_{13}+G_{23}+G_{34}
	\end{array} \right) \\ & = & 
	\left(\begin{array}{c|cc}
		60 & -35 & -25 \\\hline
		-15 & 145 & -30\\
		-45 & -10 & 110
	\end{array}\right)\,\, \mbox{mS}
\end{eqnarray*} 
Le vecteur des courants injectés aux noeuds par les sources
indépendantes d'énergie est donné par :
\[{\bf I}_{sN}=\left(
\begin{array}{r}
J\\ \hline
-J\\
0
\end{array}\right)= 
\left(\begin{array}{r}
5\\ \hline
-5\\
0
\end{array}\right)\,\, \text{A}
\]

\noindent{\bf 2. Dérivation du schéma équivalent de Norton vu de l'accès 14.}

La conductance équivalente de Norton vue de d'accès 14 s'obtient par
réduction de la matrice ${\bf G}_N$ à cet accès :
\[G_{No}=60-
\begin{pmatrix}
-35 & -25
\end{pmatrix}
\begin{pmatrix}
145 & -30\\
-10 & 110
\end{pmatrix}^{-1}
\begin{pmatrix}
-15\\
-45
\end{pmatrix}= 42.63 \mbox{~mS}\]
La source de courant de Norton équivalente est donnée par :
\[J_{No}=5-
\begin{pmatrix}
-35 & -25
\end{pmatrix}
\begin{pmatrix}
145 & -30\\
-10 & 110
\end{pmatrix}^{-1}
\begin{pmatrix}
-5\\0
\end{pmatrix}=3.69 \text{~A}\]

\pagebreak

\noindent{\bf 3. Calcul de la puissance consommée par $R$.}

Connectant la résistance  $R$ à l'accès, on déduit :

\[I=J_{No}\frac{\frac{1}{G_{No}}}{R+\frac{1}{G_{No}}}=0.7 \text{~A}\]
\[p=RI^2=49.16 \text{~W}\]

\paragraph{Exercice~\ref{ex:RSE-9}}~\\%
Le coefficient d'inductance mutuelle entre les inductances $L_1$ et
$L_4$ est donné par
\[M=k\sqrt{L_1L_4}=0.14\text{~mH}\]

On applique la méthode des mailles. L'arbre est choisi de manière à
laisser l'accès du dipôle dans un maillon. 
\begin{center}
	\includegraphics[width=0.8\linewidth]{sol_exercices/ex2-9-1}
\end{center}
Les relations de branches relatives aux branches occupées par les deux
inductances couplées s'écrivent :
\[\begin{pmatrix}
\bar{U}_a\\
\bar{U}_b
\end{pmatrix} =
\begin{pmatrix}
j\omega L_1 & j\omega M\\
j\omega M & j\omega L_4
\end{pmatrix}
\begin{pmatrix}
\bar{I}_1\\
\bar{I}_2+\bar{I}_3
\end{pmatrix}\]
Les tensions de couplage s'ajoutent vu les sens choisis pour les
courants (entrant par les points aux deux branches).

Les SLK dans chaque maille s'écrivent : 
\begin{align*}
\bar{U}& = \bar{U}_a+\bar{U}_e\\
& =j\omega L_1\bar{I}_1+j\omega M (\bar{I}_2+\bar{I}_3)+j\omega L_2(\bar{I}_1-\bar{I}_2)\\
0& = \bar{U}_c-\bar{U}_e+\bar{U}_b\\
& = j\omega L_3\bar{I}_2-j\omega L_2(\bar{I}_1-\bar{I}_2)+j\omega L_4 (\bar{I}_2+\bar{I}_3)
+j\omega M \bar{I}_1\\
0 & =\bar{U}_d+\bar{U}_b\\
& = j\omega L_5\bar{I}_3+j\omega L_4(\bar{I}_2+\bar{I}_3)+j\omega M \bar{I}_1
\end{align*}
${\bar U}$ est la tension à l'accès du dipôle.
La matrice des impédances de mailles est donnée par :
\[{\bf Z}_M=j\omega 
\begin{pmatrix}
L_1+L_2 & -L_2+M & M\\
-L_2+M & L_3+L_2+L_4 & L_4\\
M & L_4 & L_4+L_5
\end{pmatrix}=
j\omega \left[ \begin{array}{c|cc}
0.6 & -0.36 & 0.14\\ \hline
-0.36 & 1.1 & 0.4 \\
0.14 & 0.4 & 1.2
\end{array}\right]\, 10^{-3}\,\, \Omega ~.\]
La réduction de cette matrice à l'accès 1 fournit l'impédance
équivalente du dipôle, soit :
\[Z_{eq}=j\omega L_{eq}\]
\[L_{eq} =
0.6-
\begin{pmatrix}
-0.36 & 0.14
\end{pmatrix}
\begin{pmatrix}
1.1 & 0.4\\
0.4 & 1.2
\end{pmatrix}^{-1}
\begin{pmatrix}
-0.36\\
0.14
\end{pmatrix}=0.413\text{~mH}\]

\paragraph{Exercice~\ref{ex:RSE-10}}~\\%
{\bf 1. Application de la méthode des mailles}

On utilise la méthode des mailles pour déterminer le schéma équivalent
de Thévenin vu de l'accès 11'. Le graphe du circuit passifié est
représenté à la Fig.~\ref{ex2-10-1s}.
\begin{figure}[h]
	\begin{center}
		\includegraphics[width=0.45\linewidth]{sol_exercices/ex2-10-1}
		\caption{}\label{ex2-10-1s}
	\end{center}
\end{figure}

Choisissons l'arbre représenté en pointillés en prenant soin de placer
la branche 2 relative à l'accès auquel on va réduire le circuit dans
un maillon. Les mailles fondamentales correspondantes sont les
mailles I, II, III.

Adoptant les sens des courants et tensions de la
figure ci-dessous, 
\begin{center}
	\includegraphics[width=0.4\linewidth]{sol_exercices/ex2-10-2}
\end{center}
les relations de branches relatives à la paire
d'inductances couplées s'écrivent~:
\[ \left( \begin{array}{c} u_4\\ u_6 \ \end{array} \right) 
\, = \, \left( \begin{array}{ccc} L_1 && M \\ M && L_2 \end{array} \right)
\left( \begin{array}{c} \dfrac{di_4}{dt} \\ \dfrac{di_6}{dt}  \ \end{array} \right) 
+ \left( \begin{array}{ccc} 0 && 0 \\ 0 && R_2 \end{array} \right)
\left( \begin{array}{c} i_4\\ i_6 \ \end{array} \right)  \]
ou, en termes de phaseurs~:
\[ \left( \begin{array}{c} \bar{U}_4\\ \bar{U}_6 \ \end{array} \right) 
\, = \, \left( \begin{array}{ccc} j\omega L_1 && j\omega M \\ j\omega
M && R_2 + j\omega L_2 \end{array} \right) \left( \begin{array}{c}
\bar{I}_4\\ \bar{I}_6 \ \end{array} \right)~.  \] 
Le coefficient
d'inductance mutuelle \ $M$ \ est positif puisque, avec les sens
choisis pour les courants, ils entrent tous les deux par la borne
repérée par le point \ $\bullet$~.

Les courants de branches se déduisent à partir des courants de mailles~:
\[ \left( \begin{array}{c} \bar{I}_1 \\ \bar{I}_2 \\ \bar{I}_3 \end{array} \right) \]
comme indiqué à la figure ci-dessous : 
\begin{center}
	\includegraphics[width=\linewidth]{sol_exercices/ex2-10-3}
\end{center}
Les SLK pour les 3 mailles s'écrivent alors en fonction des courants de mailles~:
\begin{enumerate}
	\item maille I~:
	\begin{eqnarray*}
		\bar{E} &=& R_1 \bar{I}_1 + j\omega L_1 \, (\bar{I}_1 + \bar{I}_3) + j\omega M\, 
		(\bar{I}_1 + \bar{I}_2) +\\
		&& + \, j\omega L_2 \, (\bar{I}_1 + \bar{I}_2) + j\omega M\, (\bar{I}_1 + \bar{I}_3) + 
		R_2 \, (\bar{I}_1 + \bar{I}_2) 
	\end{eqnarray*}
	\item maille II~:
	\[ 0 \: = \: R_3\, (\bar{I}_2 - \bar{I}_3) + R_2\, (\bar{I}_1 + \bar{I}_2) + j\omega L_2 \, 
	(\bar{I}_1 + \bar{I}_2) + j\omega M \, (\bar{I}_1 + \bar{I}_3) \]
	\item maille III~:
	\[ 0 \: = \: j\omega L_1 \, (\bar{I}_1 + \bar{I}_3) + j\omega M\, 
	(\bar{I}_1 + \bar{I}_2) + R_3 \,  (\bar{I}_3 - \bar{I}_2) 
	+ R_4 \bar{I}_3~.\]
\end{enumerate}
On déduit~:
\begin{enumerate}
	\item la matrice d'impédances de mailles~:
	\[{\bf Z}_M \: = \: 
	\left( \begin{array}{c|c|c}
	R_1 + j\omega L_1 + j\omega M + j\omega L_2 & j\omega M + j\omega L_2 + R_2 & j\omega L_1 + j\omega M \\
	+ j\omega M + R_2 &&\\\hline
	\times & R_3 + R_2 + j\omega L_2 & -\, R_3 + j\omega M\\\hline
	\times &  \times & j\omega L_1 + R_3 + R_4
	\end{array} \right) \]
	matrice symétrique;
	
	\item le vecteur des f.e.m. de mailles~:
	\[ \bar{\bf V}_{sM} \: = \, \left( 
	\begin{array}{c} \bar{E} \\ 0 \\ 0 \end{array} \right) \]
\end{enumerate}
On peut aussi déterminer  \ ${\bf Z}_M$ \ par calcul à partir de l'expression
\[ {\bf Z}_M = {\bf B} \, {\bf Z}_B \, {\bf B}^T \]
avec\\
\begin{enumerate}
	\item ${\bf B}$ :  la matrice des mailles fondamentales  du graphe du circuit passifié;
	\item ${\bf Z}_B$ :  la matrice des impédances de branches.
\end{enumerate}
\[ {\bf B} =   
\begin{array}[b]{rl}
\begin{array}{r} \\ \mbox{maille} \\ \mbox{I} \\ \mbox{II} \\ \mbox{III} \end{array} & 
\begin{array}{l}
\mbox{~~~~branche}\\
\begin{array}{cccccc}
~~~~1 & 2 & 3 & 4 & ~5 & ~6 \\\hline
\end{array}\\
\left( \begin{array}{cccccc}
1 & 0 & 0 & 1 & 0 & 1\\
0 & 1 & 0 & 0 & 1 & 1\\
0 & 0 & 1 & 1 & -1 & 0
\end{array}\right)
\end{array}
\end{array}\]
\[ {\bf Z}_B \: = \, \left( \begin{array}{cccccc}
R_1 & 0 & 0 & 0 & 0 & 0\\
0 & 0 & 0 & 0 & 0 & 0\\
0 & 0 & R_4 & 0 & 0 & 0\\
0 & 0 & 0 & j\omega L_1 & 0 & j\omega M\\
0 & 0 & 0 & 0 & R_3 & 0\\
0 & 0 & 0 & j\omega M & 0 & R_2 + j\omega L_2
\end{array} \right) \]


\noindent{\bf  2. Réduction à l'accès 11' et détermination du schéma équivalent de Thévenin}

Le schéma équivalent du circuit et les accès résultant du choix de
l'arbre sont donnés à la Fig.~\ref{ex2-10-4s}.
\begin{figure}[h]
	\begin{center}
		\includegraphics[width=0.6\linewidth]{sol_exercices/ex2-10-4}
		\caption{}\label{ex2-10-4s}
	\end{center}
\end{figure}

Les relations \ $U-I$ \ pour les 3 accès s'écrivent~:
\[ \left( \begin{array}{c} 0 \\ \bar{U}_{11'} \\ 0 \end{array} \right) 
\: = \, - \left( \begin{array}{c} \bar{E} \\ 0 \\ 0 \end{array} \right) 
+ {\bf Z}_M \left( \begin{array}{c} \bar{I}_1 \\ \bar{I}_2 \\ \bar{I}_3 \end{array} \right)  \]
avec \ $\bar{U}_I = \bar{U}_{III} = 0$ \ pour les 2 accès à éliminer.

L'élimination des courants \ $\bar{I}_1\, , \, \bar{I}_3$ \ fournit la
relation cherchée à l'accès 11'~:
\[ \bar{U}_{11'} \, = \, \bar{V}_{Th} + Z_{Th} \, \bar{I}_2 \]
avec
\begin{enumerate}
	\item $\bar{V}_{Th} \, = \, 0 + {\bf Z}_{a,b} \, {\bf Z}_{b,b}^{-1} 
	\left( \begin{array}{c} \bar{E}\\ 0 \end{array} \right) $~:
	\begin{eqnarray*}
		{\bf Z}_{a,b} &=& \left( j\omega M + j\omega L_2 + R_2 ~~~-\, R_3 + j\omega M \right)\\
		{\bf Z}_{b,b} &=& \left( \begin{array}{ccc} 
			R_1 + R_2 + j\omega L_1 + j\omega L_2 + 2j\omega M && j\omega L_1 + j\omega M \\
			j\omega L_1 + j\omega M && j\omega L_1 + R_3 + R_4
		\end{array} \right)
	\end{eqnarray*}
	on trouve~: 
	\[ \bar{V}_{Th} \: = \: 85.73 + j\, 5.63 \: = \: 85.91 \angle 3.76^{\circ} \, \text{V} \]
	\item $Z_{Th} \, = \, {\bf Z}_{a,a} - {\bf Z}_{a,b} \, {\bf Z}_{b,b}^{-1} \, {\bf Z}_{b,a}$~:
	\begin{eqnarray*}
		{\bf Z}_{a,a} &=& R_3 + R_2 + j\omega L_2\\
		{\bf Z}_{b,a} &=& \left( \begin{array}{c}
			j\omega M + j\omega L_2 + R_2\\ -\, R_3 + j\omega M
		\end{array} \right) 
	\end{eqnarray*}
	on trouve~:
	\[ Z_{Th} \: = \: 33.31 + j\, 3.47 \: = \: 33.49 \angle 5.94^{\circ}\, \Omega~. \]
\end{enumerate}

\noindent{\bf 3. Réalisation de l'adaptation}

Pour soutirer au circuit une puissance maximale, il faut choisir une impédance de charge~:
\begin{eqnarray*}
	Z_L \: = \: Z_{Th}^{\ast} &=& 33.31 - j\, 3.47 \, \Omega\\
	&=& R_L - jX_L\\
	& = & R_L - \frac{j}{\omega C_L}~. 
\end{eqnarray*}
Cette impédance peut \^etre réalisée au moyen d'un circuit RC série
avec
\begin{eqnarray*}
	R_L &=& 33.31 \, \Omega\\
	\mbox{et}~~C_L &=& \frac{1}{3.47\, . \, \omega} \: = \: \frac{1}{3.47\, \times \, 10^3} 
	\: = \: 288 \, \mu \text{F~.}
\end{eqnarray*}
\noindent{\bf 4. Puissance fournie à} \ $Z_L$

Connectant $Z_L$ à l'accès 11', on dérive :

\parbox[c]{5cm}{
	\includegraphics[width=\linewidth]{sol_exercices/ex2-10-5}}
\parbox[c]{5cm}{
	\begin{eqnarray*}
		\bar{I} & = &  \dfrac{\bar{V}_{Th}}{2R_{Th}}\\
		S_{Z_L} & = & Z_L  I^2 \\
		& =&  Z_L \dfrac{V_{Th}^2}{4R_{Th}^2} \\
		&= & 55.4 - j 5.77 \,\,  \text{VA}~. 
\end{eqnarray*}}

La charge \ $Z_L$ \ consomme une puissance active 
\[ P = 55.4 \,\, \text{W} \]
et fournit une puissance réactive
\[ Q = 5.77 \,\, \text{Var~.} \]

