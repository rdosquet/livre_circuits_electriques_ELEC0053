% !TeX root = syllabus_ELEC0053.tex
% !TeX encoding = ISO-8859-1
% !TeX spellcheck = fr_FR

\section{Exercices}

\begin{exercise}{Phaseurs}
	\label{ex:phaseur_1} 
	Soient les nombres $j$, $j^2$, $j^3$ et $j^4$. 
	\begin{enumerate}
	\item Écrire les phaseurs correspondants. 
	\item Placer les sur un diagramme de phaseurs. 
	\item Quelle conclusion ce diagramme permet-il de tirer ?
	\end{enumerate}
\end{exercise}

\begin{exercise}{Impédance et admittance d'un dipôle.}\label{ex:RSE-1}
Une bobine est généralement représentée par une inductance en
série avec une faible résistance, représentative des pertes
inévitables de l'élément. Montrer, comme indiqué à la
figure ci-dessous, que l'on peut aussi repré\-sen\-ter cet élément par
la mise en parallèle d'une résistance et d'une inductance. Déterminer
l'expression des coefficients $L^{'}$ et $R^{'}$ en fonction de $L$ et
$R$.

De même, un condensateur est généralement représenté par la mise en
parallèle d'un condensateur idéal et d'une faible conductance,
représentative des pertes inévitables de l'élément. Montrer que l'on
peut aussi représenter cet élément par la mise en série d'une
conductance  et d'un condensateur idéal. Déterminer l'expression des
coefficients $C^{'}$ et $G^{'}$ en fonction de $C$ et $G$.
\begin{center}
\includegraphics[width=0.8\linewidth]{exercices/ex-2-1}
\end{center}

\rep{$R^{'}=R(1+\frac{\omega^2 L^2}{R^2})$, $L^{'}=L(1+\frac{R^2}{\omega^2 L^2})$\\
$G^{'}=G(1+\frac{\omega^2 C^2}{G^2})$, $C^{'}=C(1+\frac{G^2}{\omega^2 C^2})$}
\end{exercise}

\begin{exercise}{}\label{ex:RSE_1} 
	Déterminer les courants $\bar{I}$, $\bar{I_1}$ et $\bar{I_2}$ dans le circuit ci-dessous.
	Tracer le diagramme de phaseur et illustrer également $\bar{V}_{12}$ et $\bar{V}_{23}$.
	\begin{center}
		\includegraphics[width=0.6\textwidth]{exercices/RSE_1.pdf}
	\end{center}
\end{exercise}

\begin{exercise}{Diagramme de phaseurs.}\label{ex:RSE-2}

Dessiner un diagramme de phaseurs plausible représentant les
grandeurs indiquées sur le circuit ci-dessous. Justifier
la construction étape par étape. On prendra l'origine des phases
pour la source $\bar{E}=E\angle 0$.
\begin{center}
\includegraphics[width=0.7\linewidth]{exercices/ex-2-2}
\end{center}
\end{exercise}

\begin{exercise}{}\label{ex:RSE-3}
Une résistance de $5$ k$\Omega$ est connectée en série avec la
combinaison en parallèle d'un condensateur de capacité $C$ et d'une
inductance de 20 mH. Ce dipôle d'impédance $Z$ est inséré dans un
circuit fonctionnant en régime sinusoïdal établi à la pulsation de
1000 rad/s. On demande :
\begin{enumerate}
	\item de calculer la valeur de la capacité $C$ du condensateur pour
	que le courant parcourant le dipôle soit en retard de 45$^{\circ}$
	par rapport à la d.d.p. à ses bornes;
	\item si on connecte ce dipôle aux bornes 11$^{'}$ du circuit de la figure ci-dessous,
	dessiner le diagramme de phaseurs reprenant les tensions
	aux bornes des éléments du circuit.
\end{enumerate}
\begin{center}
\includegraphics[width=0.7\linewidth]{exercices/ex-2-3}
\end{center}

\rep{$C=49.8\,\mu$F, $\bar{U}_Z=97.25+j 16.11$ V.}
\end{exercise}


\begin{exercise}{Puissances.}\label{ex:RSE-5}
Calculer les puissances actives et réactives consommées ou
produites par les divers éléments du circuit ci-dessous. Établir les bilans de puissances
correspondants. Donner les expressions des puissances instantanées et
fluctuantes dans les divers éléments du circuit.
\begin{center}
\includegraphics[width=0.6\linewidth]{exercices/ex-2-5}
\end{center}

\rep{$S_{R_2}=3.6 + j 0$ VA, $S_{Z_1}=0.386 + j 0.002$ VA\\ $S_Z=
0.966 + j 0.966$ VA, $S_E=4.956 + j 0.973$ VA.} 
\end{exercise}

\begin{exercise}{}\label{ex:RSE-6}
Le circuit ci-dessous fonctionne en régime
sinusoïdal établi à la pulsation $\omega$. On demande~:
\begin{enumerate}
	\item de déterminer la pulsation $\omega$ pour laquelle la puissance fournie par
	le circuit à la charge $Z_L$ connectée en 11' est purement active;
	\item à cette pulsation, de calculer la puissance active fournie à la
	charge ainsi que les puissances active, réactive, instantanée et
	fluctuante fournies par la source de tension à l'instant $t= 10^{-3}\;s$.
\end{enumerate}
\begin{figure}[h]
	\begin{center}
		\includegraphics[width=0.9\linewidth]{exercices/ex-2-6}\\
		$R_1=10\mbox{~}\Omega \quad ; \quad R_2=20  \mbox{~}\Omega  \quad
		; \quad C_1=0.5  \mbox{~}\mu\mbox{F} 
		\quad ; \quad L_2=0.1  \mbox{~mH}$\\
		$R_L=5  \mbox{~}\Omega \quad ; \quad C_L=0.2 \mbox{~}\mu\mbox{F}
		\quad ; \quad L_L= 0.5 \mbox{~mH} \quad ; \quad {\bar E} = 20\angle 0 \mbox{~V (valeur efficace)}$\\
		\caption{}\label{ex2-6}
	\end{center}
\end{figure}

\rep{$\omega = 9.95\, 10^4$ rad/s, $Z_L=500\,\,\Omega$ ; 
	$P_Z = 0.575$ W, $S_E=8.3-15.5$ VA}
\end{exercise}

\begin{exercise}{}
	\label{ex:2-10}
	Un générateur délivrant une f.e.m $e(t)=E\cos \omega t$ est
	connecté, depuis en temps supposé infini, à un circuit $\cal R$
	linéaire et invariant par l'intermédiaire d'une ligne, comme
	représenté à la figure suivante. 
	\begin{center}
		\includegraphics[width=0.5\linewidth]{exercices/ex-3-10}
	\end{center}
	Cette ligne est modélisée par la
	mise en série d'une résistance $R$ et d'une inductance $L$. A
	l'instant $t=0$, un court-circuit se produit au point de connexion
	F. Déterminer l'évolution du courant $i(t)$ dans l'intervalle $t>0$.
	
	
	\rep{$i(t)=i_0e^{-t/\tau}-\frac{ E\, R}{R^2 + \omega^2 L^2}e^{-t/\tau}
		+\frac{E}{\sqrt{R^2 + \omega^2 L^2}}\cos (\omega t - \mbox{arctg}\frac{\omega L}{R})$
		A, $t\geq 0$}
\end{exercise}

\begin{exercise}{}
	\label{ex:2-1}
	Déterminer l'expression de l'inductance équivalente résultant de la mise en
	parallèle de la paire d'inductances couplées de la figure ci-dessous.
	Que devient cette inductance si la polarité magnétique de la bobine 2 est inversée ?
	\begin{center}
		\includegraphics[width=0.5\linewidth]{exercices/ex-3-1}
	\end{center}
	
	
	\rep{$L_{eq1}=\frac{L_1L_2-M^2}{L_1+L_2-2M}$\\
		$L_{eq2}=\frac{L_1L_2-M^2}{L_1+L_2+2M}$}
\end{exercise}

\begin{exercise}{}
	\label{ex:2-2}
	On considère le circuit de la figure ci-dessous.  Déterminer
	l'évolution des courants $i_0$, $i_1$, $i_2$ et de la tension $v_0$ pour
	$t\geq 0$ sachant que les deux inductances sont initialement relaxées.
	\begin{figure}[h]
		\begin{center}
			\includegraphics[width=0.85\linewidth]{exercices/ex-3-2}
		\end{center}
		\caption{}\label{ex3-2}
	\end{figure}
	Vérifier la plausibilité des résultats trouvés :
	\begin{enumerate}
		\item vérifier les valeurs initiales et finales des différentes grandeurs;
		\item vérifier les lois de Kirchhoff.
	\end{enumerate}
	
	\rep{$i_0(t)=16-16e^{-5t}$ A, $t\geq 0$\\ 
		$v_0(t)=120e^{-5t}$ V, $t\geq 0$\\
		$i_1(t)=24-24e^{-5t}$ A, $t\geq 0$\\
		$i_2(t)=-8+8e^{-5t}$ A, $t\geq 0$.}
\end{exercise}

\section{Exercices non résolus}


\begin{exercise}{}\label{ex:RSE-12}
Étant donné le circuit de la figure ci-dessous fonctionnant en régime
sinusoïdal établi à la  pulsation $\omega=1000$ rad/s, on demande :
\begin{enumerate}
	\item de déterminer l'expression des courants $i(t)$, $i_C(t)$ et
	$i_{R_2}(t)$;
	\item d'établir les bilans de puissances active et réactive;
	\item à l'instant $t=0.001$s , de calculer les puissances
	instantanées et fluctuantes consommées par la résistance $R_2$ et le
	condensateur $C$ et fournies par la source $e$.
\end{enumerate}

\begin{center}
	\includegraphics[width=0.6\linewidth]{exercices/ex-2-12}\\
	$R_1=100\,\, \Omega$, $R_2=1$, $\mbox{k}\Omega$, $C=5\,\, \mu\mbox{F}$,$L = 2 \,\, \mbox{mH} $, $	\bar{E}=10\angle 0\,\, \mbox{V} \mbox{~(valeur efficace)}$
\end{center}


\rep{$i(t)=0.06\cos (1000 t+0.94)$ , $i_C(t)=0.0589\cos(1000 t+1.14)$ , 
	$i_{R_2}(t)=0.0118\cos(1000 t-0.43)$\\
	$S_{Z_1}=0.18 + j0.004$ VA, $S_{Z_2}=0.07-j0.347$ VA\\
	$p_{R_2}=0.098$ W , $p_{fR_2}=0.029$ W\\
	$p_{C}=-0.315$ W , $p_{fC}=-0.315$ W \\
	$p_{E}=-0.166$ W , $p_{fE}=-0.416$ W}
\end{exercise}

\begin{exercise}{}\label{ex:RSE-13}
On désire modifier le circuit de l'exercice~\ref{ex:RSE-12} de manière à
ce que la puissance réactive fournie par la source $e$ soit
nulle. Pour cela, on place une impédance soit en série, soit en
parallèle avec la source $e$. Déterminer, dans chaque cas, la nature
et la valeur du ou des éléments à placer ainsi que la nouvelle
répartition des puissances actives.

\rep{$L_s= 190.3$ mH , $P_{R_1}=0.522$ W, $P_{R_2}=0.2$ W , $P_E=0.722$ W\\
	$L_{//}= 291$ mH , $P_{R_1}=0.18$ W, $P_{R_2}=0.07$ W , $P_E=0.25$ W}
\end{exercise}

\begin{exercise}{Triphasé déséquilibré étoile}
\label{ex:tri-1} 
Le circuit suivant fonctionne en 240 V triphasé. Calculez le courant $\bar{I}_c$.
\begin{center}
	\includegraphics[width=0.45\linewidth]{exercices/ex-tri-1}
\end{center}

\rep{$\bar{I}_c = -72.11 \angle 106^{\circ}$ A}
\end{exercise}

\begin{exercise}{Triphasé déséquilibré triangle}
	\label{ex:tri-2} 
	Le circuit suivant fonctionne en 450 V triphasé. Calculez le courant $\bar{I}_a$.
	\begin{center}
		\includegraphics[width=0.65\linewidth]{exercices/ex-tri-2}
	\end{center}
	
	\rep{$\bar{I}_a = 110.29 \angle -36.52^{\circ}$ A}
\end{exercise}

\begin{exercise}{Triphasé déséquilibré avec neutre}
	\label{ex:tri-3} 
	Le circuit suivant fonctionne sous une tension de ligne efficace de 240 V. Calculez les courants de ligne et le courant de neutre.
	\begin{center}
		\includegraphics[width=0.45\linewidth]{exercices/ex-tri-3}
	\end{center}
\end{exercise}

\section{Solution des exercices}
\paragraph{Exercice~\ref{ex:phaseur_1}.} On a les correspondances \\
\begin{tabular}{r r}
	$j$ & $1 e^{j\pi/2}$\\
	$j^2$ & $1 e^{j\pi}$\\
	$j^3$ & $1 e^{j3\pi/2}$\\
	$j^4$ & $1 e^0$
\end{tabular}\\
ce qui se traduit sur le diagramme de phaseurs par : 
\begin{center}
	\includegraphics[width=0.6\textwidth]{sol_exercices/phaseur_1.pdf}
\end{center}
Multiplier par $j$ revient à ajouter un déphasage de $\pi/2$.

\paragraph{Exercice~\ref{ex:RSE-1}}~\\%
\noindent{\bf 1. Schémas équivalents d'une bobine}

Les deux dip\^{o}les sont équivalents s'ils présentent la m\^{e}me
impédance \ $Z$ \ ou la m\^{e}me admittance \ $Y$.

L'impédance et l'admittance  du circuit série sont  données par :

\parbox[c]{2cm}{
	\hfill
	\includegraphics[width=0.5\linewidth]{sol_exercices/ex2-1-1}}
\parbox[c]{7cm}{
	\[ \begin{array}{rclcrcl}
	Z(j\omega ) &=& R + j\omega L 
	& \Rightarrow & Y(j\omega ) &=& \dfrac{1}{R+j\omega L} \\
	&&&&&=& \dfrac{R-j\omega L}{R^2+\omega^2 L^2}
	\end{array}\]}

L'admittance du circuit parallèle est :

\parbox[c]{2cm}{
	\hfill
	\includegraphics[width=0.9\linewidth]{sol_exercices/ex2-1-2}
	}
\parbox[c]{7cm}{
	\[Y'(j\omega ) = \dfrac{1}{R'} - \dfrac{j}{\omega L'}\]}

On déduit :
\[\begin{array}{rclcrcl}
Y(j\omega ) &=& Y'(j\omega ) 
& \Rightarrow & \dfrac{1}{R'} &=& \dfrac{R}{R^2+\omega^2 L^2}\\
&&&& R'&=& R + \dfrac{\omega^2 L^2}{R}\\
&&&&&=& R \left( 1 + \dfrac{\omega^2 L^2}{R^2} \right) \: = \: R\, (1+Q^2)
\end{array} 
\]
où \ $Q = \dfrac{\omega L}{R}$ \ est le facteur de qualité de la bobine.
\begin{eqnarray*}
	R' &=&  R\, (1+Q^2) \: \simeq \: Q^2 R ~~\mbox{si}~~Q >>\\
	L' &=& L \left( 1 + \dfrac{R^2}{\omega^2 L^2} \right) \: = \: L \left( 1 + \dfrac{1}{Q^2} \right)\\
	&\simeq&  L ~~\mbox{si}~~Q >>~
\end{eqnarray*}

\noindent{\bf 2. Schémas équivalents d'un condensateur}

Pour le circuit parallèle, on écrit :

\parbox[c]{2cm}{
	\hfill
	\includegraphics[width=0.9\linewidth]{sol_exercices/ex2-1-3}
}
\parbox[c]{7cm}{
	\begin{eqnarray*}
		Y(j\omega ) &=& G + j\omega C\\
		Z(j\omega ) &=& \dfrac{1}{G + j\omega C}\\
		&& \dfrac{G - j\omega C}{G^2 + \omega^2 C^2}
\end{eqnarray*}}

L'impédance du circuit série est :

\parbox[c]{2cm}{
	\hfill
	\includegraphics[width=0.7\linewidth]{sol_exercices/ex2-1-4}
}
\parbox[c]{7cm}{
	\[Z'(j\omega ) = \dfrac{1}{G'} - \dfrac{j}{\omega C'}\]}

On déduit :
\[ \begin{array}{rcrclcl}
Z(j\omega ) \: = \: Z'(j\omega ) 
& \Rightarrow & \dfrac{1}{G'} &=& \dfrac{G}{G^2 + \omega^2 C^2}\\
&& G'&=& G \left( 1 + \dfrac{\omega^2 C^2}{G^2} \right) &=& G \, (1 + Q^2)\\
&&&&&= & G \left( 1 + \dfrac{1}{\mbox{tg}^2 \delta} \right)
\end{array} \]
avec \ $Q$ \ le facteur de qualité du condensateur et \ $\delta$ \ son angle de pertes.
\begin{eqnarray*}
	G' &=& G\, (1+Q^2) \: \simeq \: Q^2 G ~~\mbox{si} ~~Q>>\\
	C' &=& C \left( 1 + \dfrac{G^2}{\omega^2 C^2} \right) \: = 
	\: C \left( 1 + \dfrac{1}{Q^2} \right) \: \simeq \: C ~~\mbox{si} ~~Q>>
\end{eqnarray*}

\paragraph{Exercice \ref{ex:RSE_1}.} L'impédance équivalente vaut 
\begin{align*}
Z_i &= 2+j3 + \frac{(3+j5)(5-j6)}{3+j5+5-j6} \\
&= 2+j3 + \frac{(3+j5)(5-j6)}{8-j} \\
&= 2+j3 + \frac{(5.83 \angle{59.04°} )(7.81 \angle -50.19°)}{8.06 \angle -7.13°} \\
&= 2+j3 + 5.65 \angle 15.98
&= 7.43 + j4.55 \\ &= 8.71 \angle 31.48° \Omega.
\end{align*}

\begin{align*}
\bar{I} = \frac{\bar{V}}{Z_i} = 11.48 \angle-31.48°
\end{align*}


\paragraph{Exercice~\ref{ex:RSE-2}}~\\%

\begin{center}
	\includegraphics[width=0.7\linewidth]{sol_exercices/ex2-2-1}
\end{center}

\begin{enumerate}
	\item Le courant et la tension aux bornes de la résistance  $R_2$ sont en phase.
	
	\parbox[c]{3cm}{
	\begin{center}
	\includegraphics[width=0.7\linewidth]{sol_exercices/ex2-2-2}
	\end{center}
	}
	\parbox[c]{8cm}{
		$\bar{I}_{R_2}$ \ est en phase avec \ $\bar{E}$}
	\item Le circuit aux bornes de \ $\bar{E}$ \ est de type RL~:
	
	\begin{enumerate}
		\item ~$R_2 \, // \, (R_1 + j\omega L_1 + j\omega L_2 )$ = circuit
		RL\\ $\rightarrow \: \bar{I}_E$ \ est en retard par rapport à \
		$\bar{E}$ \ d'un angle \ $0 < \varphi < \frac{\pi}{2}$~;
		
		\item ~$R_1 + j\omega L_1 + j\omega L_2 $ = circuit RL\\ $\rightarrow
		\: \bar{I}_{RL}$ \ est en retard par rapport à \ $\bar{E}$ \ d'un
		angle \ $0 < \varphi^{'} <  \frac{\pi}{2}$~.
	\end{enumerate}
	\item De la  PLK au noeud \ $A$ \ on déduit $\bar{I}_E$~:
	\[ \bar{I}_E \, = \, \bar{I}_{RL} + \bar{I}_{R_2}~ \]
	\item ~$\bar{U}_{L_2}$ \ est la tension aux bornes d'une inductance
	pure; elle est en avance de \ $\frac{\pi}{2}$ \ par rapport au courant qui la
	parcourt, c'est-à-dire \ $\bar{I}_{RL}$~
	\item La SLK fournit $ \bar{U}_{RL}$~:
	\[ \bar{U}_{RL} + \bar{U}_{L_2} \: = \: \bar{E}~ \]
\end{enumerate}

Le diagramme de phaseurs correspondant est
\begin{center}
	\includegraphics[width=0.8\linewidth]{sol_exercices/ex2-2-3}
\end{center}

\paragraph{Exercice~\ref{ex:RSE-3}}~\\%
\noindent{\bf 1. Calcul de l'impédance du dipôle}
\begin{center}
	\includegraphics[width=0.5\linewidth]{sol_exercices/ex2-3-1}
\end{center}
Pour que le courant \ $\bar{I}$ \ parcourant le dip\^{o}le présente un
retard de 45$^{\circ}$ par rapport à la tension \ $\bar{U}_Z$~ il faut
que l'argument de l'impédance \ $Z$ \ soit égal à 45$^{\circ}$.
\[\angle Z(j\omega ) = 45^{\circ}~ \]
On a~:
\begin{eqnarray*}
	Z(j\omega ) &=& R + \dfrac{1}{~j\omega C + \dfrac{1}{j\omega L}~}\\
	&=&  R + \dfrac{j\omega L}{1 -\omega^2 LC}
\end{eqnarray*}
et
\begin{eqnarray*}
	\angle Z & = & \mbox{arctg}~\dfrac{~\dfrac{\omega L}{1 - \omega^2 LC}~}{R}\\
	& = & 45^{\circ}
\end{eqnarray*}
Dès lors :
\[\dfrac{\omega L}{1 - \omega^2 LC} = R\text{~~.}\]
On déduit~:
\begin{eqnarray*}
	C &=& \dfrac{1}{\omega^2 L} \left( 1 - \dfrac{\omega L}{R} \right)\\
	&=& \dfrac{1}{10^6\, .\, 20\, 10^{-3}} 
	\left( 1 - \dfrac{10^3\, . \, 20\, 10^{-3}}{5 \, 10^3} \right) \\
	&=& 4.98\, 10^{-5} \: = \: 49.8 \,\, \mu\text{F}
\end{eqnarray*}
et
\[Z=(5+j 5)\,10^3=
7.071 \, 10^3 \angle 45^{\circ} \, \Omega~.\]

\noindent{\bf 2. Diagramme de phaseurs}\\
On connecte le dip\^{o}le \ $Z$ \ aux bornes du circuit comme indiqué ci-dessous.

\noindent\parbox[c]{8cm}{
\begin{center}
	\includegraphics[width=0.8\linewidth]{sol_exercices/ex2-3-2}
\end{center}}
\parbox[c]{3cm}{
	\begin{eqnarray*}
		Z_1 &=& R_1 + j\omega L_1\\
		&=& 2\, 10^3 + j10\\
		&=& 2\, 10^3 \angle 0.28^{\circ}
\end{eqnarray*}}

Le courant \ $\bar{I}$ \ est donné par~:
\begin{eqnarray*}
	\bar{I} \: = \: \frac{\bar{E}}{Z_1 + Z} &=& \frac{120 \angle 0}
	{2\, 10^3 + j\, 10 + 5\, 10^3 + j\, 5\, 10^3}\\
	&=& 11.3\, 10^3 - j\, 8.11 \, 10^3\\
	&=& 13.9\, 10^{3}\angle -35.6^{\circ}\, \text{A}
\end{eqnarray*}
et
\begin{eqnarray*}
	\bar{U}_{R_2} &=& 
	\bar{E} \: = \: 120\,  \angle 0\, \text{V}\\
	\bar{U}_Z &=& 
	Z\bar{I} \: = \: 97.25 + j\, 16.11 \: = \: 98.57  \angle 9.4^{\circ}\, \text{V}\\ 
	\bar{U}_{Z_1} &=& 
	Z_1\bar{I} \: = \: 22.8 - j\, 16.1 \: = \: 27.9  \angle  -35.3^{\circ}\, \text{V}
\end{eqnarray*}
On remarque qu'il existe bien un déphasage de 45$^{\circ}$ entre la
tension \ $\bar{U}_Z$ \ et le courant \ $\bar{I}$~:
\[\angle \bar{U}_Z - \angle \bar{I} 
\: = \: 9.40^{\circ} - (-35.6^{\circ}) \: = \: 45^{\circ}~. \]
Le digramme de phaseurs correspondant est 
\begin{center}
	\includegraphics[width=0.8\linewidth]{sol_exercices/ex2-3-3}
\end{center}

On vérifie~:
\[ \bar{U}_{R_2} \: = \: \bar{E} \: = \: \bar{U}_{Z_1}+ \bar{U}_Z~. \]

\paragraph{Exercice~\ref{ex:RSE-5}}~\\%
\begin{center}
	\includegraphics[width=0.7\linewidth]{sol_exercices/ex2-5-1}
\end{center}
L'état du circuit est celui déterminé à l'exercice~\ref{ex:RSE-3}~:
\begin{eqnarray*}
	\bar{I} &=& 11.3\, 10^{-3} - j\, 8.11 \, 10^{-3} \: = \: 13.9 \, 10^{-3} \angle -35.6^{\circ}\,\,
	\text{A}\\
	\bar{U}_Z &=& 98.57  \angle -9.40^{\circ}\: = \: 97.25 + j\, 16.11 \,\, \text{V}\\
	\bar{U}_{Z_1} &=& 22.8 - j\, 16.1 \: = \: 27.9  \angle -35.3^{\circ} \,\, \text{V}
\end{eqnarray*}
{\bf 1. Calcul des puissances consommées par les éléments}  \ $R\, ,\, L\, ,\, C$~.

{\em A.  Résistance } \ $R_2$ :

\begin{enumerate}
	\item puissance complexe~:
	\[ S_{R_2} \: = \: \bar{U}_{R_2} \, \bar{I}_{R_2}^{\ast} \: = \: \bar{E} \, . \, 
	\dfrac{\bar{E}^{\ast}}{R_2} \: =\: \dfrac{E^2}{R_2} \: = \: 3.6\, \text{VA~.} \]
	
	On déduit~:
	\begin{enumerate}
		\item la puissance active \ $P_{R_2} = \Re(S) = 3.6$ W
		\item et on vérifie \ $Q=0$~, une résistance ne consomme pas de puissance réactive;
	\end{enumerate}
	\item puissances instantanée et fluctuante~:
	\begin{eqnarray*}
		p(t) &=& P + |S|\, \cos\, (2\omega t \ + \phi_u + \phi_i )\\
		&=& 3.6 + 3.6 \, \cos \, 2\omega t \\
		p_f(t) &=& 3.6 \, \cos \, 2\omega t 
	\end{eqnarray*}
\end{enumerate}
{\em B.  Impédance} \ $Z_1 = R_1 + j\omega L_1$
\begin{enumerate}
	\item puissance complexe~:
	\begin{eqnarray*}
		S_{Z_1} \: = \: \bar{U}_{Z_1} \, \bar{I}^{\ast} &=& Z_1 \, \bar{I} \, \bar{I}^{\ast}\\&=& Z_1 I^2\\
		&=& R_1 I^2 + j\omega L_1I^2\\
		&=& 0.386 + j\, 0.002 \, \text{VA} 
	\end{eqnarray*}
	\item puissance active~:
	\[ P_{R_1} \, = \, R_1 I^2 \, = \, 0.386\, \text{W~.} \]
	dissipée uniquement dans \, $R_1$
	\item puissance réactive~:
	\[ Q_{L_1} \, = \, \omega L_1 I^2 \, = \, 0.002\, \text{Var} \]
	consommée par l'inductance\ $L_1~$
	\item puissances instantanée et fluctuante~:
	\begin{eqnarray*}
		p(t) &=& 0.386 + 0.386 \, \cos \, (2\omega t - 35.3^{\circ} - 35.6^{\circ})\\
		p_f(t) &=& 0.386 \, \cos \, (2\omega t - 35.3^{\circ} - 35.6^{\circ})~.
	\end{eqnarray*}
\end{enumerate}

{\em C.  Impédance} \ $Z$

A l'exercice 2.3, nous avons déduit la composition de cette impédance~:
\[ Z = (5 +j5) \, 10^3\, \Omega \]
représentée par
\begin{center}
	\includegraphics[width=0.6\linewidth]{sol_exercices/ex2-5-2}
\end{center}

\begin{enumerate}
	\item puissances complexe, active et réactive~:
	\begin{eqnarray*}
		S_Z &=& \bar{U}_Z \, \bar{I}^{\ast}\\
		&=& 0.966 + j \, 0.966\, \text{VA}\\
		P_R &=& 0.966 \, \text{W} \\
		& & \text{consommée par la résistance~}R\\
		Q_{LC} & = &0.966 \, \text{Var} \\
		& & \text{consommée par le dipôle~}LC.
	\end{eqnarray*}
	Dans ce dipôle, \ $L$ \ consomme de la puissance réactive et \ $C$ \ en produit.
	Déterminons ces puissances réactives. On a pour \ $L$~:
	\[ Q_L = \omega L\, I_L^2 \]
	avec \ $\bar{I}_L$ \ le courant parcourant l'inductance;\\
	et pour \ $C$~: 
	\[ Q_C = -\, \omega C \, U_C^2 \]
	avec \ $\bar{U}_C$ \ la tension aux bornes du condensateur \ $C$~.
	
	On calcule~:
	\begin{eqnarray*}
		\bar{U}_C &=& \bar{U}_{LC} \: = \: j5 \, 10^3 \, \bar{I} \\
		\mbox{et} ~~\bar{I}_L &=& \frac{\bar{U}_{LC}}{j\omega L} \: = \: \frac{5.10^3}{\omega L} 
		\, \bar{I} \: = \: 250\, \bar{I}~.
	\end{eqnarray*}
	Dès lors~:
	\begin{eqnarray*}
		Q_L &=& \omega LI_L^2 \: = \: \omega L(250)^2 I^2 \: = \: 241.513 \, \text{Var}\\
		Q_C &=& -\, \omega C U_C^2 \: = \, - \, \omega C \, 25\, 10^6 \, I^2 
		\: = \, -\, 240.546\, \text{Var~.}
	\end{eqnarray*}
	$L$ \ consomme \ 241.513$\,$ Var \ et \ $C$ \ produit \ 240.546$\,$ Var~. On vérifie~: 
	\[ Q_L + Q_C \, = \, Q_{LC}~. \]
\end{enumerate}

{\bf 2. Puissances fournies par la source} \ $\bar{E}$

\parbox[c]{4cm}{
	\hfill
	\includegraphics[width=0.9\linewidth]{sol_exercices/ex2-5-3}
}
\parbox[c]{8cm}{
	\begin{eqnarray*}
		\bar{I}_E &=& \bar{I} + \bar{I}_{R_2} \: = \: \bar{I} + \dfrac{\bar{E}}{R_2}\\
		&=& 41.3\, 10^{-3} - j\, 8.11 \, 10^{-3}\\
		&=& 42.1 \, 10^{-3} \angle -11.1^{\circ}
\end{eqnarray*}}

\begin{enumerate}
	\item puissances complexe, active et réactive~:
	\begin{eqnarray*}
		S_E &=& \bar{E} \, \bar{I}_E^{\ast} \: = \ 4.956 + j\, 0.973 \, \text{VA}\\
		P_E &=& 4.956 \, \text{W}\\
		Q_E &=& 0.973 \, \text{Var}
	\end{eqnarray*}
	$E$ \, produit 0.973$\,$ Var et 4.956$\,$ W
	\item puissances instantanée et fluctuante~:
	\begin{eqnarray*}
		p(t) &=& 4.956 + 5.05\, \cos \, (2\omega t - 11.1^{\circ})\\
		p_f(t) &=& 5.05 \, \cos \, (2\omega t - 11.1^{\circ})~.
	\end{eqnarray*}
\end{enumerate}
\noindent{\bf 3. Bilans de puissances}
\begin{enumerate}
	\item Puissance active~:
	\[ P_E \: = \: P_{R_1} + P_{R_2} + P_R \]
	\item Puissance réactive~:
	\[ Q_E \: = \: Q_{L_1} + Q_L + Q_C~. \]
\end{enumerate}

\paragraph{Exercice~\ref{ex:RSE-6}}~\\%
{\bf 1. Détermination de la pulsation $\omega$}

La puissance fournie à la charge $Z_L$ sera purement active si
l'impédance $Z_L$ (et donc aussi son inverse $Y_L$) est purement
réelle.
On a : 
\begin{align*}
	Z_L&=\frac{1}{j\omega C_L +\frac{1}{R_L+j\omega L_L}} \\
	Y_L & = j\omega C_L +\frac{1}{R_L+j\omega L_L}\\
	& = j\omega C_L  + \frac{R_L-j\omega L_L}{R_L^2+\omega^2 L_L^2}
\end{align*}
Il faut :
\begin{gather*}
	Im(Y_L)= 0 \Leftrightarrow 
	\omega C_L -\frac{\omega L_L}{R_L^2+\omega^2 L_L^2}=0\\
	\omega=\sqrt{\frac{1}{L_LC_L}-\frac{R_L^2}{L_L^2}} = 9.95\, 10^4 \text{~rad/s}
\end{gather*}
A cette pulsation, l'impédance $Z_L$ se réduit à la résistance :
\[Z_L=\frac{R_L^2+\omega^2L_L^2}{R_L}=500\,\, \Omega\]

{\bf 2. Puissances consommées par $Z_L$}

Dans le domaine fréquentiel à la pulsation $\omega$, le circuit
est représenté par le schéma suivant
\begin{center}
	\includegraphics[width=0.6\linewidth]{sol_exercices/ex2-6-1} 
\end{center}

Posons :
\begin{gather*}
	Z_1=\frac{1}{j\omega C_1}=-j\, 20.1 \\
	Z_2=R_2+j\omega L_2=20+j\, 9.95
\end{gather*}

Transformant la partie gauche du circuit comme suit (e.g. par équivalences de source et association en parallèle) :
\begin{center}
\includegraphics[width=\linewidth]{sol_exercices/ex2-6-2}
\end{center}
on dérive :
\[\bar{I}=\frac{\bar{E}^{'}}{Z^{'}+Z_2+Z_L}=(30.2-j\, 15.4)10^{-3}\]

La puissance complexe  consommée par la charge $Z_L$ est purement
active et est donnée par~:
\[S_L=P_L=Z_LI^2=0.575 \text{~W}\]
Détermination du courant débité par la source de tension $\bar E$ :
\begin{gather*}
	\bar{U}=(Z_L+Z_2)\bar{I}=15.9-j\, 7.73\\
	\bar{I}_E= \frac{\bar{E}-\bar{U}}{R_1}=0.415+j\, 0.773
\end{gather*}
Les puissances complexe, active et réactive  fournies par $\bar E$ sont données par :
\begin{align*}
	\text{puissance complexe :}\quad & S_E=\bar{E}\bar{I}_E^*=8.3-j \,
	15.5\text{~VA}\\
	\text{puissance active :}\quad & P_E=Re(S_E)=8.3 \text{~W}\\
	\text{puissance réactive :}\quad & Q_E=Im(S_E)=-15.5 \text{~var}
\end{align*}
La puissance fluctuante en $t=10^{-3}$ s s'exprime par :
\begin{align*}
	p_f& =|S_E|\cos(2\omega t + \angle \bar{E}+\angle \bar{I}_E)
	\text{~avec~} \angle \bar{E}=0 \quad \angle \bar{I}_E=1.078
	\text{~rad}\\
	& = 17.55\cos(2\times 9.95\, 10^4\times 10^{-3}+1.078)\\
	& = 9.72 \text{~W}
\end{align*}
On déduit la puissance instantanée en $t=10^{-3}$ s:
\[p_E=p_f+P_E=18.02 \text{~W}\]


\paragraph{Exercice~\ref{ex:2-10}}~\\%

Il faut distinguer deux périodes temporelles :
\begin{enumerate}
	\item durant la période \ $t<0$~, le circuit fonctionne en régime sinusoïdal établi;
	\item durant la période de court-circuit, \ $t\geq 0$~, le régime est
	transitoire et le circuit se transforme en celui de la
	figure ci-dessous. Il s'agit d'un simple circuit RL série.
\end{enumerate}
\begin{center}
	\includegraphics[width=0.5\linewidth]{sol_exercices/ex3-10-1}
\end{center}
Soit \ $i_0$~, la valeur du courant circulant dans le circuit à
l'instant précis du court-circuit \ $(t=0)$~.
Nous recherchons l'expression du courant $i(t)$ durant la période
$t\geq 0$ qui peut se mettre sous la forme générale
\[i(t)=i_0e^{-t/\tau}+Ae^{-t/\tau}+B\cos(\omega t + \phi )\]
avec :
\begin{enumerate}
	\item $\tau=L/R$ la constante de temps du circuit;
	\item $i_{libre}=i_0e^{-t/\tau}$ la réponse libre;
	\item $ i_{forcee}=Ae^{-t/\tau}+B\cos(\omega t + \phi )$ la réponse forcée qui comprend :
	\begin{enumerate}
		\item le terme $i_{transit}=Ae^{-t/\tau}$, partie transitoire de la réponse
		forcée, constituant avec la réponse libre la solution générale de
		l'équation différentielle homogène qui régit $i(t)$;
		\item le terme $i_{etabli}=B\cos(\omega t + \phi )$, constituant la régime
		permanent ou établi donné par la solution particulière de
		l'équation différentielle non homogène. Cette partie correspond en
		fait à la solution du circuit en régime sinosoïdal établi.  Son
		expression peut être déterminée par une analyse en phaseurs.
	\end{enumerate}
\end{enumerate}
On a,~d'après la figure
\begin{center}
	\includegraphics[width=\linewidth]{sol_exercices/ex3-10-2}
\end{center}
\[
\bar{I} = \frac{\bar{E}}{R+j\omega L}=
= \frac{\bar{E}}{\, \sqrt{R^2 + \omega ^2L^2}\, }\, e^{-j\mbox{arctg} \frac{\omega L}{R}}
\quad \text{avec} \quad \bar{E}= E \angle 0\]

et 
\[ i_{etabli}(t) \: =\: \frac{E}{\, \sqrt{R^2 + \omega ^2L^2}\, }\,
\cos \left( \omega t - \mbox{arctg} \, \frac{\omega L}{R} \right)~. \]

La condition initiale en $t=0$ impose :

\[i_0=i_0+A+ \frac{E\, R}{R^2 + \omega ^2L^2} \quad \Rightarrow
\quad A=\frac{-E\, R}{R^2+\omega^2L^2}\]
Finalement :
\[i(t)=i_0e^{-t/\tau}-\frac{E\, R}{R^2+\omega^2L^2}e^{-t/\tau}+\frac{E}{\, \sqrt{R^2 + \omega ^2L^2}\, }\,
\cos \left( \omega t - \mbox{arctg} \, \frac{\omega L}{R} \right)\]
Notons que seul le régime établi peut être calculé à partir des
phaseurs. Le régime transitoire ne peut être obtenu que par une
analyse transitoire temporelle.

\paragraph{Exercice~\ref{ex:2-1}}~\\%
a) Soient $i_1$ et $i_2$ les deux courants parcourant les 2
inductances, comme indiqué à la figure ci-dessous. Celles-ci sont
soumises à une même tension $u$ et le dipôle est alimenté par le
courant $i$.
\begin{figure}[h]
	\begin{center}
		\includegraphics[width=0.6 \linewidth]{sol_exercices/ex3-1-1}
	\end{center}
	\caption{}\label{ex3-1-1s}
\end{figure}
Etant donné les sens choisis pour les courants \ $i_1\, , \, i_2$~,
entrant tous deux par la borne repérée par le point \ $\bullet$~, on
écrit~:
\begin{eqnarray*}
	u_1 &=& u \: = \: L_1 \dfrac{di_1}{dt} + M \dfrac{di_2}{dt}\\
	u_2 &=& u \: = \: M \dfrac{di_1}{dt} + L_2 \dfrac{di_2}{dt}
\end{eqnarray*}
On déduit la relation suivante entre \ $i_1$ et $i_2$~:
\begin{eqnarray*}
	L_1 \dfrac{di_1}{dt} + M \dfrac{di_2}{dt} &=&  M \dfrac{di_1}{dt} + L_2 \dfrac{di_2}{dt}\\
	\dfrac{di_1}{dt} &=& \dfrac{L_2-M}{L_1-M} \, \dfrac{di_2}{dt}
\end{eqnarray*}
D'autre part, la PLK écrite au noeud 1 fournit~:
\[ i\,=\; i_1+i_2 \]
et donc
\[ \dfrac{di}{dt} \: = \: \dfrac{di_1}{dt} + \dfrac{di_2}{dt}~. \]
On dérive successivement~:
\begin{eqnarray*}
	u &=& \left( L_1 \dfrac{L_2-M}{L_1-M} + M \right)  \dfrac{di_2}{dt}\\
	\dfrac{di}{dt} &=& \left( \dfrac{L_2-M}{L_1-M} + 1 \right)  \dfrac{di_2}{dt} 
\end{eqnarray*}
et finalement, la relation \ $u-i$ \ aux bornes du dipôle s'écrit~:
\begin{eqnarray*}
	u &=& \left( L_1 \dfrac{L_2-M}{L_1-M} + M \right) \dfrac{1}{\dfrac{L_2-M}{L_1-M} + 1} \,
	\dfrac{di}{dt}\\
	&=& \dfrac{L_1L_2 - M^2}{L_1+L_2-2M} \,  \dfrac{di}{dt}~. 
\end{eqnarray*}
Le dipôle est donc représenté par une inductance équivalente~:
\[ L_{eq} \: = \: \dfrac{L_1L_2-M^2}{L_1+L_2-2M}~. \]

b) Si la polarité magnétique de la bobine 2 est inversée comme
indiqué à la figure suivante,
\begin{center}
	\includegraphics[width=0.6\linewidth]{sol_exercices/ex3-1-2}
\end{center}
on écrit~: 
\begin{eqnarray*}
	u &=& L_1 \dfrac{di_1}{dt} - M \dfrac{di_2}{dt}\\
	u &=& -\, M  \dfrac{di_1}{dt} + L_2  \dfrac{di_2}{dt}~.
\end{eqnarray*}
On dérive successivement~:
\begin{eqnarray*}
	\dfrac{di_1}{dt} &=& \dfrac{L_2+M}{L_1+M} \, \dfrac{di_2}{dt}\\
	\dfrac{di}{dt} &=& \left( \dfrac{L_2+M}{M_1+M} + 1 \right) \dfrac{di_2}{dt}\\
	u &=& \left( L_1 \dfrac{L_2+M}{L_1+M} - M \right) \dfrac{di_2}{dt}
\end{eqnarray*}
et finalement~:
\begin{eqnarray*}
	u &=& \left( L_1 \dfrac{L_2+M}{L_1+M} - M \right)   \, \dfrac{1}{L_1 \frac{L_2+M}{L_1+M} + 1} 
	\,  \dfrac{di}{dt}\\
	&=& \dfrac{L_1L_2 - M^2}{L_1 + L_2 + 2M} \,  \dfrac{di}{dt}~.
\end{eqnarray*}
L'inductance équivalente du dipôle est donc donnée par~:
\[ L_{eq} \: = \: \dfrac{L_1L_2 - M^2}{L_1 + L_2 + 2M}~. \]


\paragraph{Exercice~\ref{ex:2-2}}~\\%

\noindent{\bf 1. Expression de \ $i_0$}\\
Les deux inductances couplées, connectées en parallèle, peuvent être
remplacées par l'inductance équivalente (voir exercice 3.1)~:
\[ L_{eq} \: = \: \dfrac{L_1L_2 - M^2}{L_1+L_2 -2M} 
\: = \: \dfrac{45-36}{18-12} \: = \: 1.5 \, \mbox{H}~. \]
On aboutit donc au circuit \ $RL$ \ suivant : 
\begin{center}
	\includegraphics[width=0.5\linewidth]{sol_exercices/ex3-2}
\end{center}
La constante de temps de ce circuit est~:
\[ \tau \, = \, \dfrac{L_{eq}}{R} \, = \, \dfrac{1.5}{7.5} \, = \, 0.2\, \mbox{s}~. \]
La courant d'établissement de ce circuit est donné par~:
\[ i_0(t) \: = \: \dfrac{E}{R} \left( 1 - e^{-t/\tau}\right) \, = \: 16-16\, e^{-5t}
\, \mbox{A}~~,~~t\geq 0\]
étant donné qu'il est supposé qu'aucune énergie n'est initialement stockée dans l'inductance 
\ $(i_0(0) = 0)$~.

\newpage

\noindent{\bf 2. Expression de \ $v_0$}\\
La tension \ $v_0$ \ s'obtient en appliquant la SLK dans la maille. On déduit~:
\begin{eqnarray*}
	v_0 &=& E - Ri_0\\
	&=& 120 - 7.5\, i_0\\
	&=& 120 \, e^{-5t} \, \mbox{V}~~,~~t \geq 0^+~. 
\end{eqnarray*}

\noindent{\bf 3. Expression de \ $i_1$ et $i_2$}\\
Les courants \ $i_1$ et $i_2$ \ se déduisent des relations de branches
relatives aux deux inductances couplées. Ainsi, on a~:
\[ v_0 \: = \: L_1 \dfrac{di_1}{dt} + M \dfrac{di_2}{dt} 
\: = \: M \dfrac{di_1}{dt} + L_2 \dfrac{di_2}{dt} \]
soit
\begin{eqnarray*}
	3 \dfrac{di_1}{dt} + 6\dfrac{di_2}{dt} &=&  6\dfrac{di_1}{dt} + 15 \dfrac{di_2}{dt}\\
	\mbox{ou}~~~~~~\dfrac{di_1}{dt} &=& -\, 3 \dfrac{di_2}{dt}~.
\end{eqnarray*}
On a aussi~: 
\begin{eqnarray*}
	i_0 &=& i_1+i_2\\
	\mbox{ou}~~~~~~\dfrac{di_0}{dt} &=& \dfrac{di_1}{dt} + \dfrac{di_2}{dt}\\
	&=& -\, 2 \dfrac{di_2}{dt}~.
\end{eqnarray*}
Remplaçant $\dfrac{di_0}{dt}$ par son expression, on dérive :
\begin{eqnarray*}
	80\,e^{-5t} & =& -2\, \dfrac{di_2}{dt}\\
	\mbox{soit~~~~} i_2(t) &=& \int^t_0 - 40\, e^{-5x}\, dx\\
	&=& -\, 8 + 8\, e^{-5t}\, \text{A}~~, ~~t\geq 0
\end{eqnarray*}
et 
\begin{eqnarray*}
	i_1(t) &=& i_0(t) - i_2(t)\\
	&=& 24 - 24\,  e^{-5t}\, \text{A}~~, ~~t\geq 0~.
\end{eqnarray*}

\newpage

\noindent{\bf 4. Vérification des valeurs limites}\\
a) \ $i_0$~:
\begin{center}
	\begin{tabular}{lp{8cm}}
		$t=0^+$~: & $i_0(0^+) = 0$~\\ &  on a bien \ $i_0(0^+) = i_0(0^-)$,
		continuité du courant dans une inductance.\\ 
		$t=\infty$~: &
		${\displaystyle \lim_{t\rightarrow\infty}}\, i_0(t) = 16 =
		{\displaystyle \dfrac{E}{R}}$\\ 
		& lorsque le régime est établi,
		l'inductance ne joue plus aucun rôle; elle ne s'oppose plus au passage
		du courant.
	\end{tabular}
\end{center}

\noindent b) \ $v_0\, , \, i_1\, , \, i_2$~:
\begin{enumerate}
	\item Remarquons tout d'abord que les expressions trouvées pour \
	$i_1$ et $i_2$ \ sont compatibles avec celles de \ $v_0$~.
	
	On vérifie en effet que~:
	\begin{eqnarray*}
		v_0 &=& 3 \dfrac{di_1}{dt} + 6 \dfrac{di_2}{dt} \: = \: 360\, e^{-5t} - 240\, e^{-5t}\\
		&=& 120 \, e^{-5t} \, \text{V}~~,~~ t\geq 0^+
	\end{eqnarray*}
	ou que
	\begin{eqnarray*}
		v_0 &=& 6 \dfrac{di_1}{dt} + 15 \dfrac{di_2}{dt} \: = \: 720\, e^{-5t} - 600\, e^{-5t}\\
		&=& 120 \, e^{-5t} \, \text{V}~~,~~ t\geq 0^+~.
	\end{eqnarray*}
	
	\item Les valeurs finales de \ $i_1$ et $i_2$ \ peuvent être vérifiées
	en utilisant les flux totaux embrassés par chacune des deux bobines~:
	\begin{enumerate}
		\item pour la bobine 1~: 
		\begin{eqnarray*}
			\phi_1 &=& L_1i_1 + Mi_2\\
			&=& 3i_1 + 6i_2 
		\end{eqnarray*}
		\item pour la bobine 2~: 
		\begin{eqnarray*}
			\phi_2 &=& Mi_1 + L_2i_2\\
			&=& 6i_1 + 15i_2 ~~.
		\end{eqnarray*}
	\end{enumerate}
	On trouve ~:
	\[ \phi_1 \: = \: \phi_2 \: = \: 24 - 24\, e^{-5t}~~\mbox{Wb}~. \]
	On remarque que l'on a bien~:
	\[ v_0 \, = \, \dfrac{d\phi_1}{dt} \, = \, \dfrac{d\phi_2}{dt} 
	\, = \, 120\, e^{-5t} \, \text{V}~~, ~~t\geq 0^+~. \]
	Les valeurs finales des flux sont~: 
	\[ \phi_1(\infty ) = \phi_2(\infty ) = 24~\mbox{Wb}~. \]
	Ces valeurs sont compatibles avec les valeurs finales des courants. On vérifie en effet~: 
	\begin{eqnarray*}
		i_1(\infty ) \: = \: \lim_{t\rightarrow\infty} \left( 24-24\, e^{-5t} \right) \: = \: 24~\mbox{A}\\
		i_2(\infty ) \: = \: \lim_{t\rightarrow\infty} \left( -8+8\, e^{-5t} \right) \: = \, - 8~\mbox{A}
	\end{eqnarray*}
	et
	\begin{eqnarray*}
		\phi_1(\infty )  &=& L_1i_1(\infty ) + Mi_2(\infty ) \: 
		= \: 3\times 24 + 6\times (-8) \: = \: 24~\mbox{Wb}\\
		\phi_2(\infty )  &=& Mi_1(\infty ) + L_2i_2(\infty ) \: 
		= \: 6\times 24 + 15\times (-8) \: = \: 24~\mbox{Wb}
	\end{eqnarray*}
\end{enumerate}

Il faut remarquer qu'il n'était pas possible de vérifier directement les valeurs finales de \ $ i_1$ et $i_2$ \ sans recourir au calcul des flux. En effet, pour \ $t\rightarrow\infty$, les 2 bobines se comportent comme
des court-circuits, et il n'est pas possible de déterminer comment le courant total de c.c, $i_0$, \ se divise entre les deux bobines.


